\documentclass[12pt, a4paper, oneside]{article}
\usepackage{CJKutf8}
\usepackage{amsmath, amsthm, amssymb, bm, color, framed, graphicx, hyperref, mathrsfs}
\usepackage{fancyhdr}
\usepackage[left=20mm, right=20mm, top=20mm, bottom=20mm]{geometry}
\pagestyle{fancy}
\fancyhf{}
\rhead{\begin{CJK}{UTF8}{gbsn}{计83 李天勤 2018080106}\end{CJK}}
\lhead{\begin{CJK}{UTF8}{gbsn}{初等数论及应用作业1}\end{CJK}}
	

\title{\textbf{课程作业}}
\author{Dylaaan}
\date{\today}
\newpage
\linespread{1.5}
\definecolor{shadecolor}{RGB}{241, 241, 255}
\newcounter{problemname}
\newenvironment{problem}{\begin{shaded}\stepcounter{problemname}\par\noindent\textbf{题目\arabic{problemname}. }}{\end{shaded}\par}
\newenvironment{solution}{\par\noindent\textbf{解答. }}{\par}
\newenvironment{note}{\par\noindent\textbf{题目\arabic{problemname}的注记. }}{\par}

\begin{document}
\begin{CJK}{UTF8}{gbsn}

% \maketitle
% \newpage

\section{1.2 最大公因子和小公倍数}

\setcounter{problemname}{3}
\begin{problem}
    设$a,b,c\in \mathbb{Z},  a\neq 0$, 则$a \mid bc$ 当且仅当 $\frac{a}{(a,b)}\mid c$
\end{problem}


\begin{problem}
    $m$和$n$是互素的正整数:证明:
    \newline 
    $(1)$ 对于每个整数 $a,(a,mn)=(a,m)(a,n)$ \newline
    $(2)$ $mn$的每个正因子$d$均可惟一地表示成$d=d_1d_2$, 其中 $d_1$和$d_2$分别为$m$和$n$地正因子。
\end{problem}

\begin{problem}
    设$n$为正整数,$a,b$是不全为零的整数,证明:\newline
    $(1) (a^n,b^n)=(a,b)^n$ \newline
    $(2)$ 若$a$和$b$是互素的正整数,$ab=c^n, c\in \mathbb{Z}$,则$a$和$b$都是正整数的$n$次方幂,事实上,$a=(a,c)^n, b= (b,c)^n$ 
\end{problem}


\setcounter{problemname}{8}
\begin{problem}
    用辗转相除法求963和957地最大公因子,并求出方程$963x + 657y = (963, 957)$ 的全部整数解
\end{problem}

\begin{problem}
    求下列方程的全部整数解 \newline
    $(1) 6x + 20y - 15x = 23$ \newline
    $(2) 25x + 13y + 7x = 2$ 
\end{problem}


\setcounter{problemname}{11}
\begin{problem}
    设$f(x)=x^n+a_1x^n-1 + \dots + a_{n-1}x+a_n$ 是首项系数为1的整系数多项式,则$f(x)$的每个有理数必为整数
\end{problem}


\begin{problem}
    说$m$和$n$为正整数,则在$n,2n,\dots,mn$这$m$个数当中恰有$(m,n)$个是$m$的倍数。
\end{problem}

\setcounter{problemname}{15}
\begin{problem}
    设$m$和$n$是互素的非零整数,证明:对每个整数$a$,如果$m\mid a, n\mid a$,则 $mn \mid a$
\end{problem}


\section{1.3 惟一分解定理}

\setcounter{problemname}{2}
\begin{problem}
    设$a,b,c$均为正整数,证明
\end{problem}

\begin{problem}
    整数$n$叫作无平方因子,是指不存在整数$m\ge2$,是的$m^2\mid n$, 证明
\end{problem}

\end{CJK}
\end{document}