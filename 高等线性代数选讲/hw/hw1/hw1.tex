\documentclass[12pt, a4paper, oneside]{article}
\usepackage{CJKutf8}
\usepackage{amsmath, amsthm, amssymb, bm, color, framed, graphicx, hyperref, mathrsfs}
\usepackage{fancyhdr}
\usepackage[left=20mm, right=20mm, top=20mm, bottom=20mm]{geometry}
\pagestyle{fancy}
\fancyhf{}
\setlength{\headheight}{14.60191pt}
\rhead{\begin{CJK}{UTF8}{gbsn}{计83 李天勤 2018080106}\end{CJK}}
\lhead{\begin{CJK}{UTF8}{gbsn}{高等线性代数选讲作业1}\end{CJK}}
	

\title{\textbf{课程作业}}
\author{Dylaaan}
\date{\today}
\newpage
\linespread{1.5}
\definecolor{shadecolor}{RGB}{241, 241, 255}
\newcounter{problemname}
\newenvironment{problem}{\begin{shaded}\stepcounter{problemname}\par\noindent\textbf{题目\arabic{problemname}. }}{\end{shaded}\par}
\newenvironment{solution}{\par\noindent\textbf{解答. }}{\par}
\newenvironment{note}{\par\noindent\textbf{题目\arabic{problemname}的注记. }}{\par}
\newcommand{\TAG}[1]{\tag*{$\left\tagldelim\parbox{\tagwidth}{#1}\right\tagrdelim$}}

\begin{document}
\begin{CJK}{UTF8}{gbsn}

% \maketitle
% \newpage

\section{Chapter 1: Topics in Linear Algebra}

\begin{problem}
We would like to find a real $n \times n$ matrix $A$ such that $A^2 - I$ \newline
$1.$ For each even number n, find a real solution \newline
$2.$ If odd $n$, show that there is no real solution
\end{problem}

\begin{solution}

  We can use an example to prove that when $n$ is even, there exists a matrics whose square equals the negative identity. 
  For example, the matrix, 
  \begin{equation}
    A = \begin{bmatrix}
      0 & -1 \\ 
      1 & 0 
    \end{bmatrix}
  \end{equation}
  To prove that no real solution exists when $n$ is odd, we can use proof by contradiction.
  The key in solving this problem is using the determinant of the matrices. \newline
  Let $A, B$ be $n \times n$ matrices and $c$ be scalar. Then we have \newline
  $1. \det{(AB)} = \det{(A)}\det{(B)}$ \newline
  $2. \det{(cA)} = c^n\det(A)$ \newline
  Assuming that we have $A$ such that $A^2=-I$, we know that if two matrices are similar, then their determiant is equal, thus we obtain 
  $$ \det(A^2) = \det(-I) $$
  $$ \det(A^2) = (-1)^n \det(I) = -1 $$ 
  because $n$ is odd and $\det(I) = 1$ \newline
  Since $A$ is a real matrix, the determinant of $A$ is also real. Thus, the solution to $\det(A)^2=-1$ is impossible. 
  Hence there is no such real $A$ that satisfies the assumption.

\end{solution}

% \begin{note}
%     这里是注记. 
% \end{note}

\begin{problem}
  Suppose $A^2 = -I$ for a real $n \times n$ matrix $A$. For each vector $v \in \mathbb{R}^n$, we write $iv$ to mean $Av$. 
  For any $n \times n$ matrix $B$, we say it is complex linear if $B(kv)=kBv$ for any complex number $k\in \mathbb{C}$
\end{problem}

\begin{solution}
  If $A^2$ = -I. That means $A$ is composed with itself on the identity map on $\mathbb{R}$. That means that the effect of applying
  $A$ twice is the same as multiplication by -1. This is similar to multiplication by imaginary unit $i$.
  Thus complex scalar multiplication can be defined as 
  $$ (x + iy)v = xv + yA(v) $$ 
  1. In order to prove that B is complex linear if and only if AB=BA, we have to prove two things. \newline
  (a) When $B$ is complex linear, $\Rightarrow AB=BA$ \newline
  (b) If $AB=BA$, then $B$ is complex linear. \newline
  Proof (a): We know that $B$ is complex linear, therefore $\forall k \in \mathbb{C}$ 
  $$ B(kv) = kB(v) $$
  $$ \Rightarrow B((a+bi)v) = a+bi(Bv) $$
  $$ \Rightarrow \forall a,b \in \mathbb{R}, aBv + bB(iv) = a + bi(Bv)$$ 
  $$ \Rightarrow \forall a,b \in \mathbb{R}, aBv + bBAv = a + bA(Bv)$$
  Proof (b): \newline
  2. Let $A$ = 
  $$ A = \begin{bmatrix}
    0 & -1 \\ 1 & 0 
  \end{bmatrix}, X = \begin{bmatrix}
    1 & 2 \\ -1 & -1
  \end{bmatrix}$$ 
  Since we know that property $AB=BA$, we can show that $X$ does not have to be complex linear

  3. We can pick $C_1, C_2$ such that
  $$ C_1 = \begin{bmatrix}
    1 & 0 \\ 0 & -1
  \end{bmatrix},  C_2 = \begin{bmatrix}
    0 & -1 \\ -1 & 0
  \end{bmatrix}
  $$
 
\end{solution}

\begin{problem}
  If $V$ is an abstract vector space over $C$, then for each vector $v$ and each $k \in C$, obviously $kv$ is well defined.
  But as a result, for each vector $v$ and each $k\in \mathbb{R}$, $kv$ must be well-defined. 
  So any complex vector space must be a real vector space (but NOT vice versa). 
\end{problem}

\begin{solution}
  \newline
  1. The map $C$ is real linear. 
  $$ \forall k, a, b \in \mathbb{R}, c = a + bi$$
  $$ k = i, kc' = b + ai, (kc)' = -b-ai, kc' \neq (kc)' $$
  $$ \Rightarrow c(kv) \neq k(cv)$$
 Therefore map cannot be complex linear and only real linear. But, can be real linear. \newline 
  2. $\mathbb{C}$ linear implies $\mathbb{R}$ linear because if $C$ is linear, then when $C = a$, and $a\in \mathbb{R}$   \newline 
  3. $\mathbb{R}$ basis = $\begin{bmatrix}
    1 \\ 0
  \end{bmatrix}, \begin{bmatrix}
    0 \\ 1
  \end{bmatrix}, \begin{bmatrix}
    i \\ 0
  \end{bmatrix}, \begin{bmatrix}
    0 \\ i
  \end{bmatrix} $, real dimension 4
  \newline
  $\mathbb{C}$ basis = $\begin{bmatrix}
    1 \\ 0
  \end{bmatrix}, \begin{bmatrix}
    0 \\ 1
  \end{bmatrix}
  $, complex dimension 2
  \newline
  4. $\mathbb{C}$ linear independence implies $\mathbb{R}$ linear independence because
  $$\forall c_1, \dots, c_k \in \mathbb{C}, \sum{(a_j+b_ji)v} = 0, \therefore a_j = 0, j = 1, \dots, k $$
  \newline
  5. $\mathbb{C}$ spanning implies $\mathbb{R}$ spanning because the basis of the real span is a subset of the complex span
\end{solution}



\begin{problem}
  Adapted from Gilbert Strang 9.3.11-15. Take the permutation matrix
\end{problem}

\begin{solution}
  A permutation matrix simply manipulates the rows. Sending 
  $\begin{bmatrix}
    a \\ b \\ c\\ d 
  \end{bmatrix}$
  to 
  $\begin{bmatrix}
    b \\ c \\ d \\ a
  \end{bmatrix}$
  would require permutation matrix 
  $$ 
    P = \begin{bmatrix}
      0 & 1 & 0 & 0 \\
      0 & 0 & 1 & 0 \\ 
      0 & 0 & 0 & 1 \\ 
      1 & 0 & 0 & 0
    \end{bmatrix}
  $$
  And $F_4$ Fourier matrix is 
  $$ 
    F_4 = \begin{bmatrix}
      1 & 1 & 1 & 1 \\
      1 & i & -1 & -i  \\ 
      1 & -1& 1 & -1 \\
      1 & -i & -1 & i
    \end{bmatrix}
  $$

  1. 
  $$ P\begin{bmatrix}
    1 \\ 1 \\ 1 \\ 1 
  \end{bmatrix} 
  = \begin{bmatrix}
    1 \\ 1 \\ 1 \\ 1 
  \end{bmatrix} ,
  P\begin{bmatrix}
    1 \\ i \\ -1 \\ -i
  \end{bmatrix} =
  \begin{bmatrix}
    i \\ -1 \\ -i \\ 1
  \end{bmatrix}
  $$
  2. $PF_4 = F_4D$ 
  
  $$
  \begin{bmatrix}
    0 & 1 & 0 & 0 \\
    0 & 0 & 1 & 0 \\ 
    0 & 0 & 0 & 1 \\ 
    1 & 0 & 0 & 0
  \end{bmatrix} 
  \begin{bmatrix}
    1 & 1 & 1 & 1 \\
    1 & i & -1 & -i  \\ 
    1 & -1& 1 & -1 \\
    1 & -i & -1 & i
  \end{bmatrix}
  = 
  \begin{bmatrix}
    1 & i & -1 & -i  \\ 
    1 & -1& 1 & -1 \\
    1 & -i & -1 & i \\
    1 & 1 & 1 & 1 \\
  \end{bmatrix}
  =
  \begin{bmatrix}
    1 & 1 & 1 & 1 \\
    1 & i & -1 & -i  \\ 
    1 & -1& 1 & -1 \\
    1 & -i & -1 & i
  \end{bmatrix}
  \begin{bmatrix}
    a_1 &&&\\
    & a_2 &&\\
    && a_3 &\\
    &&& a_4\\
  \end{bmatrix}
  $$
  where $ a_1 = 1, a_2 = i, a_3 = -1, a_4 = -i$ \newline
  Eigenvalues and eigen vectors of $P$\newline
  The characteristic equation is  $(P - \lambda I) = 0$
  Where we get that $\det(P - \lambda I)$ is 
  $$
    \det\left(\begin{bmatrix}
      -\lambda & 1 & 0 & 0 \\
      0 & -\lambda & 1 & 0 \\ 
      0 & 0 & -\lambda & 1 \\
      1 & 0 & 0 & -\lambda  
    \end{bmatrix}\right) = \lambda^4 - 1 = 0
  $$
  Thus the solutions $\lambda_1 = 1, \lambda_2 = -1, \lambda_3 = i, \lambda_4 = -i$ \newline
  Sub into the characteristic polynomial and solve using reduced row echelon form to get eigen vectors. 
  The eigenvectors for $\lambda_1,\lambda_2,\lambda_3,\lambda_4$ are
  $$ 
    \begin{bmatrix}
      1 \\ 1 \\ 1 \\ 1
    \end{bmatrix}, 
    \begin{bmatrix}
      -1 \\ 1 \\ -1 \\ 1 
    \end{bmatrix},
    \begin{bmatrix}
      i \\ -1 \\ -i \\ 1
    \end{bmatrix},
    \begin{bmatrix}
      -i \\ -1 \\ i \\ 1
    \end{bmatrix}
  $$
  respectively. \newline
  3. 
  $$ 
    C = \begin{bmatrix}
      c_0 & c_1 & c_2 & c_3 \\
      c_3 & c_0 & c_1 & c_2 \\ 
      c_2 & c_3 & c_0 & c_1 \\ 
      c_1 & c_2 & c_3 & c_0
    \end{bmatrix} \begin{bmatrix}
      1 \\ 1 \\ 1 \\ 1 
    \end{bmatrix} = 
    \begin{bmatrix}
      c_0 + c_1 + c_2 + c_3 \\
      c_0 + c_1 + c_2 + c_3 \\
      c_0 + c_1 + c_2 + c_3 \\
      c_0 + c_1 + c_2 + c_3 
    \end{bmatrix}
  $$ 

  $$ 
  C = \begin{bmatrix}
    c_0 & c_1 & c_2 & c_3 \\
    c_3 & c_0 & c_1 & c_2 \\ 
    c_2 & c_3 & c_0 & c_1 \\ 
    c_1 & c_2 & c_3 & c_0
  \end{bmatrix} \begin{bmatrix}
    1 \\ i \\ -1 \\ -i 
  \end{bmatrix} =
  \begin{bmatrix}
    c_0 + ic_1 - c_2 - ic_3 \\
    c_3 + ic_0 - c_1 - ic_2 \\ 
    c_2 + ic_3 - c_0 - ic_1 \\ 
    c_1 + ic_2 - c_3 - ic_0
  \end{bmatrix}
  $$ 
  4. Write $C$ as a polynomial of $P$. Find the eigenvalues and eigen vectors of $C$
  $$
  P = \begin{bmatrix}
    0 & 1 & 0 & 0 \\
    0 & 0 & 1 & 0 \\
    0 & 0 & 0 & 1 \\
    1 & 0 & 0 & 0
  \end{bmatrix},
  P^2 = \begin{bmatrix}
    0 & 0 & 1 & 0 \\
    0 & 0 & 0 & 1 \\
    1 & 0 & 0 & 0 \\
    0 & 1 & 0 & 0
  \end{bmatrix},
  P^3 = \begin{bmatrix}
    0 & 0 & 0 & 1 \\
    1 & 0 & 0 & 0 \\
    0 & 1 & 0 & 0 \\
    0 & 0 & 1 & 0
  \end{bmatrix}, 
  P^4 = \begin{bmatrix}
    1 & 0 & 0 & 0 \\
    0 & 1 & 0 & 0 \\
    0 & 0 & 1 & 0 \\
    0 & 0 & 0 & 1
  \end{bmatrix}
  $$

  $$ C = f(p) = C_0P^4 + C_1P + C_2P^2 +C_3P^3 $$
  $$ \lambda_C = C_0 + C_1\lambda + C_2\lambda^2 + C_3\lambda^3 $$
  $$ \lambda_{C_1} = C_0 + C_1 + C_2  + C_3, x_1 = \begin{bmatrix}
    1 \\ 1 \\ 1 \\ 1
  \end{bmatrix}$$
  $$ \lambda_{C_1} = C_0 - iC_1 - C_2  + iC_3, x_2 = \begin{bmatrix}
    1 \\ -i \\ -1 \\ i
  \end{bmatrix} $$
  $$ \lambda_{C_1} = C_0 - C_1 - C_2  - C_3, x_3 = \begin{bmatrix}
    1 \\ -1 \\ 1\\ -1
  \end{bmatrix} $$
  $$ \lambda_{C_1} = C_0 + iC_1 - C_2  - iC_3, x_4 = \begin{bmatrix}
    1 \\ i \\ 1\\ -i
  \end{bmatrix} $$



\end{solution}


\end{CJK}
\end{document}