\documentclass[10pt, a4paper, oneside]{article}
\usepackage{CJKutf8}
\usepackage{amsmath, amsthm, amssymb, bm, color, framed, graphicx, hyperref, mathrsfs}
\usepackage{fancyhdr}
\usepackage{multicol}
\usepackage{mathtools}
\usepackage{gauss}
\usepackage[left=20mm, right=20mm, top=20mm, bottom=20mm]{geometry}
\pagestyle{fancy}
\fancyhf{}
\rhead{\begin{CJK}{UTF8}{gbsn}{计83 李天勤 2018080106}\end{CJK}}
\lhead{\begin{CJK}{UTF8}{gbsn}{高代线性代数作业4}\end{CJK}}
	

\title{\textbf{课程作业}}
\author{AkaCoder404}
\date{\today}
\newpage
\linespread{1.5}
\definecolor{shadecolor}{RGB}{241, 241, 255}
\newcounter{problemname}
\newenvironment{problem}{\begin{shaded}\stepcounter{problemname}\par\noindent\textbf{题目\arabic{problemname}. }}{\end{shaded}\par}
\newenvironment{solution}{\par\noindent\textbf{解答. }}{\par}
\newenvironment{note}{\par\noindent\textbf{题目\arabic{problemname}的注记. }}{\par}

% row operations
\newenvironment{sysmatrix}[1]
 {\left[\begin{array}{@{}#1@{}}}
 {\end{array}\right]}
\newcommand{\ro}[1]{%
  \xrightarrow{\mathmakebox[\rowidth]{#1}}%
}
\newlength{\rowidth} % row operation width

% kbordermatrix




% begin document  
\begin{document}
\begin{CJK}{UTF8}{gbsn}

% \maketitle
% \newpage

\section{1.4 Minimal Polynomials, Slyvester's Equation}


\setcounter{problemname}{0}
\begin{problem}
  Considering the matrix $A = \begin{bmatrix}
    1 & 2 & 1 & 2 \\ 
    0 & 1 & 3 & 4 \\
    0 & 0 & 3 & 5 \\
    0 & 0 & 0 & 4
  \end{bmatrix}$ \newline
\end{problem}

\begin{solution} \newline
    $(1)$ Find a matrix $B$ such that $BAB^{-1} = \begin{bmatrix}
      1 & 2 & 0 & 0 \\ 
      0 & 1 & 0 & 0 \\
      0 & 0 & 3 & 5 \\
      0 & 0 & 0 & 4
    \end{bmatrix}$ \newline
    Let $A = \begin{bmatrix}
      A_{11} & A_{12} \\ 
      0 & A_{22}
    \end{bmatrix}$, thus $BAB^{-1} = \begin{bmatrix}
      A_{11} & 0 \\ 
      0 & A_{22}
    \end{bmatrix}$ \newline Let
    $$ B = \begin{bmatrix}
      I & X \\ 0 & I 
    \end{bmatrix}, B^{-1} = \begin{bmatrix}
      I & -X \\ 0 & I 
    \end{bmatrix}$$
    $$\therefore BAB^{-1} = \begin{bmatrix}
      I & X \\ 0 & I 
    \end{bmatrix} \begin{bmatrix}
      A_{11} & A_{12} \\ 
      0 & A_{22}
    \end{bmatrix} \begin{bmatrix}
      I & -X \\ 0 & I 
    \end{bmatrix} $$ 
    $$ \Rightarrow \begin{bmatrix}
      A_{11} & A_{12} + XA_{11} - XA_{22} \\ 0 & A_{22}
    \end{bmatrix}$$ Therefore
    $$ A_{12} + XA_{11} - XA_{22} = 0$$
    $$ A_{12} = XA_{22} - XA_{11} \space \text{(sylvestor's equation)}$$
    $A_{11}$ and $A_{22}$ have no common eigenvalues, thus, there exists a unique solution $X = \begin{bmatrix}
      a & b \\ c & d
    \end{bmatrix}$ such that 
    $$\begin{bmatrix}
      -2a + 2c & 2d-5a-3b \\ 
      -2c & -5c-3d 
    \end{bmatrix} = \begin{bmatrix}
      1 & 2 \\ 3 & 4 
    \end{bmatrix} = \begin{cases}
      a = -2 \\
      b = 31/9 \\
      c = -3/2 \\
      d = 7/6
    \end{cases}$$
    $$\therefore B = \begin{bmatrix}
      I & X \\ 
      0 & I
    \end{bmatrix} = \begin{bmatrix}
      1 & 0 & -2 & \frac{31}{9} \\
      0 & 1  & -\frac{3}{2} & \frac{7}{6} \\
      0 & 0 & 1 & 0 \\
      0 & 0 & 0 & 1
    \end{bmatrix}$$
    $(2)$ Find a basis for the subspace $V_3 + V_4$, where $V_\lambda$ is the eigenspace of $A$ for the eigenvalue $\lambda$ \newline
    The eigenvalue of $BAB^{-1}$ is equal to the eigenvalue of $A$
    $$ BAB^{-1} - 3I  = \begin{bmatrix}
      -2 & 2 & 0 & 0 \\
      0 & -2 & 0 & 0 \\
      0 & 0 & 0 & 5 \\
      0 & 0 & 0 & 1 
    \end{bmatrix}$$
    Eigenvector $\vec{v_1} = \begin{pmatrix}
      0 \\ 0 \\ 1 \\ 0
    \end{pmatrix}$, $V_3 = \text{span}\left\{ \begin{pmatrix}  0 \\ 0 \\ 1 \\ 0\end{pmatrix}\right\}$  
    $$ BAB^{-1} - 4I  = \begin{bmatrix}
      -3 & 2 & 0 & 0 \\
      0 & -3 & 0 & 0 \\
      0 & 0 & -1 & 5 \\
      0 & 0 & 0 & 0
    \end{bmatrix}$$
    Eigenvector $\vec{v_2} = \begin{pmatrix}
      0 \\ 0 \\ 5 \\ 1
    \end{pmatrix}$, $V_4 = \text{span}\left\{ \begin{pmatrix}  0 \\ 0 \\ 5\\ 1\end{pmatrix}\right\}$ \newline
    $\because BAB^{-1}v = \lambda v \Rightarrow A(B^{-1}v)=\lambda(B^{-1}v))$ \newline
    $\vec{v'_1} = B^{-1}\vec{v_1} = \begin{pmatrix}
      2 \\ \frac{3}{2} \\ 1 \\ 0
    \end{pmatrix}$,  
    $\vec{v'_2} = B^{-1}\vec{v_2} = \begin{pmatrix}
      \frac{59}{9} \\ \frac{19}{3} \\ 5 \\ 1
    \end{pmatrix}$ \newline
    $ \therefore $ Basis for subspace $V_3 + V_4$ is 
    $\left\{ 
      \begin{pmatrix} 2 \\ \frac{3}{2} \\ 1 \\ 0 \end{pmatrix}     
      \begin{pmatrix} \frac{59}{9} \\ \frac{19}{3} \\ 5 \\ 1 \end{pmatrix}
    \right\} $
\end{solution}

\begin{problem}
  Suppose we have complex matrix $A = \begin{bmatrix}
    B & I \\ & B
  \end{bmatrix}$. We know the characteristic polynomial of $A$ is just the square of the character polynomial of $B$. 
  Is the minimal polynomial of $A$ the square of minimal polynomial of $B$
\end{problem}

\begin{solution}
  $(1)$ $\begin{bmatrix}
    X & 0 & I & 0 \\
    0 & Y & 0 & I \\
    0 & 0 & X & 0 \\
    0 & 0 & 0 & Y
  \end{bmatrix}$ \newline 

  \[ \begin{gmatrix}[b]
    X & 0 & I & 0 \\
    0 & 0 & X & 0 \\
    0 & Y & 0 & I \\
    0 & 0 & 0 & Y
  \rowops
  \swap{1}{2} 
  \end{gmatrix}
  \]

  \[ \begin{gmatrix}[b]
    X & I & 0 & 0 \\
    0 & X & 0 & 0 \\
    0 & 0 & Y & I\\
    0 & 0 & 0 & Y
    \colops
    \swap{1}{2}
  \end{gmatrix}
  \]therefore
  $$\begin{bmatrix}
    1 & 0 & 0 & 0 \\
    0 & 0 & 1 & 0 \\
    0 & 1 & 0 & 0 \\
    0 & 0 & 0 & 1
  \end{bmatrix} \begin{bmatrix}
    X & 0 & I & 0 \\
    0 & Y & 0 & I \\
    0 & 0 & X & 0 \\
    0 & 0 & 0 & Y
  \end{bmatrix}\begin{bmatrix}
    1 & 0 & 0 & 0 \\
    0 & 0 & 1 & 0 \\
    0 & 1 & 0 & 0 \\
    0 & 0 & 0 & 1
  \end{bmatrix} = \begin{bmatrix}
    X & I & 0 & 0 \\
    0 & X & 0 & 0 \\
    0 & 0 & Y & I \\
    0 & 0 & 0 & Y
  \end{bmatrix} $$, proving that the two matrices are similar \newline
  $(2)$ $B = \begin{bmatrix}
    3 & 0 \\ 0 & 4
  \end{bmatrix}, B^2 = \begin{bmatrix}
    9 & 0 \\ 0 & 16
  \end{bmatrix}$ \newline
  Let $B$'s minimal polynomial be $f(x) = a_2x^2 + a_1x + a_0$ \newline
  $\begin{cases}
    9a_2 + 3a_1 + a_0 = 0 \\
    16a_2 + 4a_1 + a_0 = 0
  \end{cases} \Rightarrow \begin{cases}
    a_2 = \frac{1}{2}a_0 \\
    a_1 = -\frac{7}{12}a_0
  \end{cases}$ \newline
  Therefore $B's$ minimal polynomial is $f(x) = x^2 - 7x + 12$ \newline
  $A = \begin{bmatrix}
    3 & 0 & 1 & 0 \\
    0 & 4 & 0 & 1 \\
    0 & 0 & 3 & 0 \\
    0 & 0 & 0 & 4
  \end{bmatrix}$
  $\therefore$ $A$ is similar to $XY$ in $(1)$, therefore, we know that $A$ is similar to the matrix, 
  $$\begin{bmatrix}
    3 & 0 & 1 & 0 \\
    0 & 3 & 0 & 0 \\
    0 & 0 & 4 & 1 \\
    0 & 0 & 0 & 4 
  \end{bmatrix}$$
  which is also a Jordan Canonical Form of $A$. Thus, the minimal polynomial of $A$ is $f(x) = (x-3)^2(x-4)^2 = x^4 - 14x^3 + 73x^2 - 168x + 144$ \newline
  $(3)$ \newline
  $B = \begin{bmatrix}
    0 & 1 \\ 0 & 0 
  \end{bmatrix}, A = \begin{bmatrix}
    0 & 1 & 1 & 0 \\
    0 & 0 & 0 & 1 \\
    0 & 0 & 0 & 1 \\
    0 & 0 & 0 & 0
  \end{bmatrix}, \text{rank}(A) = 2$ \newline
  $A^2 = \begin{bmatrix}
    B^2 & 2B \\ 0 & B^2 
  \end{bmatrix}, A = \begin{bmatrix}
    0 & 0 & 0 & 2 \\
    0 & 0 & 0 & 0 \\
    0 & 0 & 0 & 0 \\
    0 & 0 & 0 & 0
  \end{bmatrix}, \text{rank}(A^2) = 1$ \newline
  $A^3 = \begin{bmatrix}
    B^3 & 2B^2 \\ 0 & B^3 
  \end{bmatrix}, A = \begin{bmatrix}
    0 & 0 & 0 & 0 \\
    0 & 0 & 0 & 0 \\
    0 & 0 & 0 & 0 \\
    0 & 0 & 0 & 0
  \end{bmatrix}, \text{rank}(A^3) = 0$ \newline
  $A^4 = 0$, therefore the size of $A$'s Jordan blocks are $1\times 1, 3\times 3$ \newline
  $J = \begin{bmatrix}
    0 & 1 & 0 & 0 \\
    0 & 0 & 1 & 0 \\ 
    0 & 0 & 0 & 0 \\
    0 & 0 & 0 & 0
  \end{bmatrix}$ \newline 
  $(4)$ $B = \begin{bmatrix}
    0 & 1 \\ 0 & 0
  \end{bmatrix}, \therefore$ the minimal polynomial for $B$ is $f(x) = x^2$ \newline
  $\Rightarrow A = \begin{bmatrix}
    B & I \\ 0 & B 
  \end{bmatrix} = \begin{bmatrix}
    0 & 1 & 1 & 0 \\
    0 & 0 & 0 & 1 \\ 
    0 & 0 & 0 & 1 \\
    0 & 0 & 0 & 0 
  \end{bmatrix}$ \newline
  $\Rightarrow J.C.F = \begin{bmatrix}
    0 & 1 & 0 & 0 \\
    0 & 0 & 1 & 0 \\
    0 & 0 & 0 & 0 \\
    0 & 0 & 0 & 0
  \end{bmatrix}, \therefore$ the minimal polynomial for $A$ is $f(x) = x^3$ \newline
  $(5)$ When $B = \begin{bmatrix}
    3 & 0 \\ 0 & 4 
  \end{bmatrix}, \min{B} = x^2 - 7x + 12, \min{A} = x^4 - 14x^3 +73x^2 - 168x + 144$ \newline
  When $B = \begin{bmatrix}
    3 & 0 \\ 0 & 4 
  \end{bmatrix}, \min{B} = x^2, \min{A} = x^3
  $ \newline
  Thus, when matrix $B$ is non-singular, the minimal polynomial of matrix $A$ is the (minimal of polynomial of $B)^2$



\end{solution}

\begin{problem}
  In class we see that for Sylvester's equations $AX-XB=C$, if $A$, $B$, have no common eigenvalue, then there is always a unique solution. 
  What if $A$, $B$ have common eigenvalues? \newline
  Let us take an extreme case, and assume that $A=B$. \newline
  So we are looking at an equation $AX-XA=C$ for constant $n \times n$ matrices $A$, $C$.  \newline
  Let $V$ be the space of $n\times n$ matrices and consider the linear map $L:V\rightarrow V$ such that $L(X) = AX-XA$
\end{problem}

\begin{solution} \newline
  $(1)$ $X = kA, k \in N^*$ has infinite solutions for $L(X) = 0$ \newline
  $(2)$ $L(XY) = AXY - XYA$ \newline 
  $\Rightarrow L(X)Y + XL(Y) = (AX -XA)Y + X(AY-YA)$ \newline
  $\Rightarrow AXY - XAY + XAY - XYA$  \newline
  $\Rightarrow AXY - XYA$
  $\therefore L(XY) = L(X)Y + XL(Y)$ \newline 
  $(3)$ [DON'T KNOW]\newline
  $(4)$ If $p(X)$ is $X$'s minimal polynomial, when $L(X) = I$, then $L(p(X)) = L(X)p'(X) = Ip'(X)=p'(X)$ \newline
  $\therefore p'(X)=0$, which contradicts $p(X)$ being $X$'s minimal polynomial, thus $L(X) = I$ has no solutions \newline
  $(5)$ Let $A = PBP^{-1}, B = \text{diag}(\lambda_1,\dots,\lambda_n), AX -XA = 0 \Rightarrow PBP^{-1}X - XPBP^{-1} = 0$ \newline
  $\Rightarrow BP^{-1}X - P^{-1}XPBP = 0$ \newline
  $\Rightarrow BP^{-1}XP - P^{-1}XPB = 0$ \newline  
  Let $Y = P^{-1}XP, BY-YB = 0$
  $$ Y = \begin{bmatrix}
    y_{11} & y_{12} & \cdots & y_{1n} \\
    \vdots &  & & \vdots \\
    y_{n1} & \cdots & \cdots & y_{nn}
  \end{bmatrix}
  $$
  $$ \begin{bmatrix}
    \lambda_1 y_{11} & \lambda_1 y_{12} & \cdots & \lambda_1 y_{1n} \\
    \vdots &  & & \vdots \\
    \lambda_n y_{n1} & \cdots & \cdots &   \lambda_n y_{nn}
  \end{bmatrix} = \begin{bmatrix}
    \lambda_1 y_{11} & \lambda_2 y_{12} & \cdots & \lambda_n y_{1n} \\
    \vdots &  & & \vdots \\
    \lambda_1 y_{n1} & \cdots & \cdots &   \lambda_n y_{nn}
  \end{bmatrix} = $$ Thus, the free elements exist on on the diagonal line. \newline
  $\therefore \dim{\ker{L}} = n$
  
  $(6)$ Let $A$ = $(a_{ij}), i\leq 3, j \leq 3, X = (x_{ij}), i \leq 3, j \leq 3$\newline 
  The diagonal of $AX = \begin{cases}
    a_{11}x_{11} + a_{12}x_{21} + a_{13}x_{31} \\
    a_{21}x_{12} + a_{22}x_{22} + a_{23}x_{32} \\
    a_{31}x_{13} + a_{32}x_{23} + a_{33}x_{33}
   \end{cases}$ \newline
  Diagonal of $XA = \begin{cases}
    a_{11}x_{11} + a_{21}x_{12} + a_{31}x_{13} \\
    a_{12}x_{21} + a_{22}x_{22} + a_{32}x_{23} \\
    a_{13}x_{31} + a_{23}x_{32} + a_{33}x_{33} \end{cases}
  $ \newline
  $\begin{cases}
    a_{11}x_{11} + a_{12}x_{21} + a_{13}x_{31} =  a_{11}x_{11} + a_{21}x_{12} + a_{31}x_{13} \\
    a_{21}x_{12} + a_{22}x_{22} + a_{23}x_{32} =  a_{12}x_{21} + a_{22}x_{22} + a_{32}x_{23} \\
    a_{31}x_{13} + a_{32}x_{23} + a_{33}x_{33} =  a_{13}x_{31} + a_{23}x_{32} + a_{33}x_{33} 
  \end{cases}$ \newline

  $\Rightarrow \begin{cases}
    a_{12} = a_{21} = a_{31} = a_{13} = 0 \\
    a_{21} = a_{12} = a_{23} = a_{32} = 0 \\
    a_{13} = a_{31} = a_{23} = a_{32} = 0 \\
  \end{cases}$ \newline
  $A = \begin{bmatrix}
    1 & 0 & 0 \\ 0 & 2 & 0 \\ 0 & 0 & 3
  \end{bmatrix}$




\end{solution}

\end{CJK}
\end{document}