\documentclass[12pt, a4paper, oneside]{article}
\usepackage{CJKutf8}
\usepackage{amsmath, amsthm, amssymb, bm, color, framed, graphicx, hyperref, mathrsfs}
\usepackage{fancyhdr}
\usepackage{multicol}
\usepackage{mathtools}
\usepackage{gauss}
\usepackage[left=10mm, right=10mm, top=20mm, bottom=20mm]{geometry}
\pagestyle{fancy}
\fancyhf{}
\rhead{\begin{CJK}{UTF8}{gbsn}{计83 李天勤 2018080106}\end{CJK}}
\lhead{\begin{CJK}{UTF8}{gbsn}{高代线性代数作业8}\end{CJK}}
	

\title{\textbf{课程作业}}
\author{AkaCoder404}
\date{\today}
\newpage
\linespread{1.5}
\definecolor{shadecolor}{RGB}{241, 241, 255}
\newcounter{problemname}
\newenvironment{problem}{\begin{shaded}\stepcounter{problemname}\par\noindent\textbf{题目\arabic{problemname}. }}{\end{shaded}\par}
\newenvironment{solution}{\par\noindent\textbf{解答. }}{\par}
\newenvironment{note}{\par\noindent\textbf{题目\arabic{problemname}的注记. }}{\par}

% row operations
\newenvironment{sysmatrix}[1]
 {\left[\begin{array}{@{}#1@{}}}
 {\end{array}\right]}
\newcommand{\ro}[1]{%
  \xrightarrow{\mathmakebox[\rowidth]{#1}}%
}
\newlength{\rowidth} % row operation width

% kbordermatrix

% begin document  
\begin{document}
\begin{CJK}{UTF8}{gbsn}

% \maketitle
% \newpage
\section{Multilinear Maps}

\setcounter{problemname}{0}
\begin{problem}
  Elementary Layer Operations for Tensors
\end{problem}

\begin{solution}
  1. \newline
\end{solution}

\begin{solution}
  2. \newline
\end{solution}

\begin{solution}
  3. \newline
\end{solution}

\begin{solution}
  4. \newline
\end{solution}

\begin{solution}
  5. \newline
\end{solution}


%--------------------------------------------------------------------
\begin{problem}
  Let $M$ be a $3\times 3\times 3$ "3D matrix" whose $(i,j,k)$ entry is $i+j+k$. We interpret this as a multilinear map $M: \mathbb{R}^3\times\mathbb{R}^3\times\mathbb{R}^3\rightarrow \mathbb{R}$
\end{problem}

\begin{solution}
  1. \newline
\end{solution}

\begin{solution}
  2. \newline
\end{solution}

\begin{solution}
  3. \newline
\end{solution}

\begin{solution}
  4. \newline
\end{solution}

\end{CJK}
\end{document}