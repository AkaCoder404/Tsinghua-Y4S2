\documentclass[12pt, a4paper, oneside]{article}
\usepackage{CJKutf8}
\usepackage{amsmath, amsthm, amssymb, bm, color, framed, graphicx, hyperref, mathrsfs}
\usepackage{fancyhdr}
\usepackage{multicol}
\usepackage{mathtools}
\usepackage{gauss}
\usepackage[left=10mm, right=10mm, top=20mm, bottom=20mm]{geometry}
\pagestyle{fancy}
\fancyhf{}
\rhead{\begin{CJK}{UTF8}{gbsn}{计83 李天勤 2018080106}\end{CJK}}
\lhead{\begin{CJK}{UTF8}{gbsn}{高代线性代数作业8}\end{CJK}}
	

\title{\textbf{课程作业}}
\author{AkaCoder404}
\date{\today}
\newpage
\linespread{1.5}
\definecolor{shadecolor}{RGB}{241, 241, 255}
\newcounter{problemname}
\newenvironment{problem}{\begin{shaded}\stepcounter{problemname}\par\noindent\textbf{题目\arabic{problemname}. }}{\end{shaded}\par}
\newenvironment{solution}{\par\noindent\textbf{解答. }}{\par}
\newenvironment{note}{\par\noindent\textbf{题目\arabic{problemname}的注记. }}{\par}

% row operations
\newenvironment{sysmatrix}[1]
 {\left[\begin{array}{@{}#1@{}}}
 {\end{array}\right]}
\newcommand{\ro}[1]{%
  \xrightarrow{\mathmakebox[\rowidth]{#1}}%
}
\newlength{\rowidth} % row operation width

% kbordermatrix

% begin document  
\begin{document}
\begin{CJK}{UTF8}{gbsn}

% \maketitle
% \newpage
\section{Multilinear Maps}

\setcounter{problemname}{0}
\begin{problem}
  Elementary Layer Operations for Tensors
\end{problem}

\begin{solution}
  1. \newline
  $(i,j,k)$ entry of 3D matrix $\alpha \otimes \beta \otimes \lambda$ 
  \begin{align}
    \alpha \otimes \beta \otimes \lambda &= \alpha \otimes \beta \otimes \lambda \left( \begin{pmatrix}
      0 \\ \vdots  \\ i \\ \vdots \\ 0
    \end{pmatrix}, \begin{pmatrix}
      0 \\ \vdots  \\ j \\ \vdots \\ 0
    \end{pmatrix}, \begin{pmatrix}
      0 \\ \vdots  \\ k\\ \vdots \\ 0
    \end{pmatrix}\right)\nonumber \\
    & = \left[\alpha 
    \begin{pmatrix}
      0 \\ \vdots  \\ 1 \\ \vdots \\ 0
    \end{pmatrix}i,  
    \right]\left[\beta
    \begin{pmatrix}
      0 \\ \vdots  \\ 1 \\ \vdots \\ 0
    \end{pmatrix}j,  
    \right]\left[\lambda
    \begin{pmatrix}
      0 \\ \vdots  \\ 1 \\ \vdots \\ 0
    \end{pmatrix}k
    \right] \nonumber \\
     &= \alpha i\beta j \lambda k \nonumber 
  \end{align}
\end{solution}

\begin{solution}
  2. \newline
  All 3D matrices are sum of $\lambda$ rank-one 3D matrices \newline
  $$A = \alpha_1 \otimes \beta_1 \otimes \lambda_1 + \alpha_2 \otimes  \beta_2 \otimes \lambda_3 + \cdots + \alpha_r \otimes \beta_r \otimes \lambda_r $$
  $$ M_3 : A \mapsto (\alpha_1E) \otimes \beta_1 \otimes \lambda_1 + (\alpha_2 E)\otimes \beta_2 \otimes \lambda_2 + \cdots + (\alpha_r E)\otimes \beta_r \otimes \lambda_r $$ 
  $$ \alpha_1(E_1 + E_2)\otimes B_1 \otimes \lambda_1 + \alpha_2(E_1+E_2)\otimes \beta_2 \otimes \lambda_2 + \cdots + \alpha_r(E_1 +E_2)\otimes\beta_r\otimes\lambda_r $$ 
  $$ = \alpha_1E_1\otimes\beta_1\otimes\lambda_1+\alpha_2E_1\otimes\beta_2\otimes\lambda_2+\cdots+\alpha_rE_1\otimes\beta_r\otimes\lambda_r$$
  $$ + \alpha_1E_2\otimes\beta_1\otimes\lambda_1+\alpha_2E_2\otimes\beta_2\otimes\lambda_2+\cdots+\alpha_rE_2\otimes\beta_r\otimes\lambda_r$$
  $\because$
  $$\alpha_1(kE)\otimes\beta_1\otimes\lambda_1+\alpha_2(kE)\otimes\beta_2\otimes\lambda_2+\cdots+\alpha_r(kE)\otimes\beta_r\otimes\lambda_r$$
  $$=k\left[\lambda_1E\otimes\beta_1\otimes\lambda_1+\alpha_2E\otimes\beta_2\otimes\lambda_2+\cdots+\alpha_r\alpha_E\otimes\beta_r\otimes\lambda_r\right]$$
  $\therefore$ $M_E$ is a linear map and layer operation is extended to a linear map $M_E$
\end{solution}

\begin{solution}
  3. \newline
  Let $A$ be a rank $r$ 3D tensor, $B=M_E(A)$ \newline
  We need to prove that $rank(B) = r$
  Since $A$ is a rank $r$ 3D tensor, $A$ can be expressed as 
  $$A = \alpha_1\otimes\beta_1\otimes\lambda_1 + \alpha_2\otimes\beta_2\otimes\lambda_2+\cdots+\alpha_r\otimes\beta_r\lambda_r$$
  $$B = M_E(A) = (\alpha_1E)\otimes\beta_1\otimes\lambda_1 + (\alpha_2(E))\otimes\beta_2\otimes\lambda_2+\cdots+(\alpha_rE)\otimes\beta_r\lambda_r$$
  $\therefore$ $rank(B)=rank(A)$ and $rank(B)\leq r$ \newline
  Thus, if $rank(B) = p < r$, then $B=\alpha_1\otimes\beta_1\otimes\lambda_1+\cdots+\lambda_p\otimes\beta_p\otimes\lambda_p$ \newline
  Since $B=M_E(A)$, then $A = M_E^{-1}(B)$
  $$A=(\lambda_rE^{-1}\otimes\beta_1\otimes\lambda_1+\cdots+\alpha_pE^{-1}\otimes\beta_p\otimes\lambda_p)$$
  where $rank(A) \leq p < r$, which is contradicting \newline
  $\therefore$ $rank(B) = r$
\end{solution}

\begin{solution}
  4. \newline
  If a 3D matrix has rank $\lambda_1, \lambda_1 < r$, then it can be expressed as the sum of $r_1$ rank-one tensor \newline
  $$ M = \alpha_1\otimes\beta_1\otimes\lambda_1 + \alpha_2\otimes\beta_2\otimes\lambda_2 + \cdots + \alpha_{r_1}\otimes\beta_{r_1}\otimes\lambda_{r_1}$$
  So, the $i^{th}$ value of the horizontal layer is 
  $$ M_i = \lambda_{1i}(\alpha_1\otimes\beta_1) + \lambda_{2i}(\alpha_2\otimes\beta_2)+\cdots+\lambda_{r_1i}(\alpha_n\otimes\beta_{r1})$$ 
  $\therefore$ \newline
  All layers of $M$ can expressed as sum of $r_1$ rank one matrix \newline 
  $rank(M_i) \leq r_1 < r \forall i $ \newline
  Which contradicts our known conditions, therefore 3D matrix has at least rank r 
\end{solution}

\begin{solution}
  5. \newline
  A 2D layer matrix has rank 2, $rank(M) \geq 2$
  $$ \begin{bmatrix}
    1 & 0 \\ 0 & 1
  \end{bmatrix} = \begin{bmatrix}
      \frac{1}{2} & -\frac{1}{2} \\
      -\frac{1}{2} & \frac{1}{2}
  \end{bmatrix} + \begin{bmatrix}
    \frac{1}{2} & \frac{1}{2} \\
    \frac{1}{2} & \frac{1}{2}
  \end{bmatrix}$$

  $$ \begin{bmatrix}
    0 & 1 \\ 1 & 0
  \end{bmatrix} = \begin{bmatrix}
    -\frac{1}{2} & \frac{1}{2} \\
     \frac{1}{2} & -\frac{1}{2}
  \end{bmatrix} + \begin{bmatrix}
    \frac{1}{2} & \frac{1}{2} \\
    \frac{1}{2} & \frac{1}{2}
  \end{bmatrix}$$
  $$ M = \begin{pmatrix}
      \frac{1}{2} \\ -\frac{1}{2}
  \end{pmatrix} \otimes \begin{pmatrix}
    1 \\ -1
  \end{pmatrix} \otimes \begin{pmatrix}
    1 \\ -1
  \end{pmatrix} + \begin{pmatrix}
    \frac{1}{2} \\ \frac{1}{2} 
  \end{pmatrix} \otimes \begin{pmatrix}
    1 \\ 1
  \end{pmatrix}\otimes \begin{pmatrix}
    1 \\ 1
  \end{pmatrix}$$
  $\because rank(M) \leq 2 \therefore rank(M)=2$
\end{solution}


%--------------------------------------------------------------------
\begin{problem}
  Let $M$ be a $3\times 3\times 3$ "3D matrix" whose $(i,j,k)$ entry is $i+j+k$. We interpret this as a multilinear map $M: \mathbb{R}^3\times\mathbb{R}^3\times\mathbb{R}^3\rightarrow \mathbb{R}$
\end{problem}

\begin{solution}
  1. \newline
  Horizontal layer 1 of $M$ is $A_1 = \begin{bmatrix}
    3 & 4 & 5 \\ 
    4 & 5 & 6 \\
    5 & 6 & 7 \\
  \end{bmatrix}$ \newline
  Horizontal layer 2 of $M$ is $A_2 = \begin{bmatrix}
    4 & 5 & 6 \\
    5 & 6 & 7 \\
    6 & 7 & 8
  \end{bmatrix}$ \newline
  Horizontal layer 3 of $M$ is $A_3 = \begin{bmatrix}
    5 & 6 & 7 \\ 
    6 & 7 & 8 \\ 
    7 & 8 & 9 
  \end{bmatrix}$ \newline
  Since $M$ sends $(u,v,w)$ to $\left[u^TA_1v,u^TA_2v,u^TA_3v\right]w$ \newline
  So $M$ sends $(u,v,w)$ to $\left[v^TA_1v,v^TA_2v,v^TA_3v\right]v$ \newline
  $v^TA_1v=(x,y,z)\begin{pmatrix}
    3 & 4 & 5 \\ 
    4 & 5 & 6 \\
    5 & 6 & 7
  \end{pmatrix} \begin{pmatrix}
    x \\ y \\ z
  \end{pmatrix} = 3x^2 +5y^2+7z^2+8xy+10xz+12yz$ \newline
  $v^TA_2v=(x,y,z)\begin{pmatrix}
    4 & 5 & 6 \\
    5 & 6 & 7 \\
    6 & 7 & 8 
  \end{pmatrix}\begin{pmatrix}
    x \\ y \\ z
  \end{pmatrix} = 4x^2+6y^2+8z^2+10xy+12xz+14yz$ \newline
  $v^TA_3v=(x,y,z)\begin{pmatrix}
    5 & 6 & 7 \\ 
    6 & 7 & 8 \\ 
    7 & 8 & 9 
  \end{pmatrix}\begin{pmatrix}
    x \\ y \\ z
  \end{pmatrix} = 5x^2+7y^2+9z^2+12xy+14xz+16yz$ \newline
  $\therefore$
  \begin{align}
    \left[v^TA_1v, v^TA_2v, v^TA_3v\right]\begin{pmatrix}
      x \\ y \\ z
    \end{pmatrix} &= x(3x^2+5y^2+7z^2+8xy+10xz+12yz) \nonumber \\
    & + y(4x^2+6y^2+8z^2+10xy+12xy+14yz)\nonumber \\
    & + z(5x^2+7y^2+9z^2+12xy+14xz+16yz)\nonumber 
  \end{align}
  $$ = 3x^5+6y^5+9z^3+12x^2y+15xy^2+21xz^2+15x^2z+22yz^2+21y^2z +36xyz $$
\end{solution}

\begin{solution}
  2. \newline
  $M: (v_1, v_2, v_3) \mapsto M(v_1, v_2, v_3)$ \newline
  $M': (v_1, v_2, v_3) \mapsto M'(\sigma(v_1), \sigma(v_2), \sigma(v_3))$ \newline
  $M'$ is attained by exchanging the $x,y,z$ direction of $M$, follwoing the order of the $\sigma$ map. \newline
  Since we know the entry of $M'$, 
  $$ f(i) + f(j) + f(k)  = i + j + k$$ 
  $\therefore$, the $(i,j,k)$ entry of the multilinear map $M^{\sigma}$ is always equal to $i+j+k$ \newline
  $i\in \{1,2,3\}, j\in \{1,2,3\}, k\in \{1,2,3\}$ \newline
  $M^{\sigma}$ is always equal to $M$ when $\sigma$ is an identity map \newline 
  $M(v_1, v_2, v_3) = M(v_{\sigma(1)}, v_{\sigma(2)}, v_3{\sigma(3)})$
\end{solution} 

\begin{solution}
  3. \newline
  Horizontal layer 1 of $M$, $A_1=\begin{bmatrix}
    3 & 4 & 5 \\ 4 & 5 & 6 \\ 5 & 6 & 7
  \end{bmatrix}$ \newline
  $rank(A_1) = 2$ \newline
  From the previous answer we know $rank(M)\geq rank(A_1)=2$ \newline
  \begin{align}
    M &= \begin{pmatrix}
      4 & 5 & 6 \\ 
      5 & 6 & 7 \\
      6 & 7 & 8
    \end{pmatrix} \otimes \begin{pmatrix}
      1 \\ 1 \\ 1
    \end{pmatrix} + \begin{pmatrix}
      1 & 1 & 1 \\ 1 & 1 & 1 \\ 1 & 1 & 1
    \end{pmatrix} \otimes \begin{pmatrix}
      -1 \\ 0 \\ 1
    \end{pmatrix} \nonumber \\
    & = \left[\begin{pmatrix}
      1 \\ 1 \\ 1
    \end{pmatrix}\otimes\begin{pmatrix}
      5 \\ 6 \\ 7
    \end{pmatrix}+\begin{pmatrix}
      1 \\ 1 \\ 1
    \end{pmatrix}\otimes\begin{pmatrix}
      -1 \\ 0 \\ 1
    \end{pmatrix}\right]\otimes\begin{pmatrix}
      1 \\ 1 \\ 1
    \end{pmatrix} + \begin{pmatrix}
      1 \\ 1 \\ 1
    \end{pmatrix}\otimes\begin{pmatrix}
      1 \\ 1 \\ 1
    \end{pmatrix}\otimes\begin{pmatrix}
      -1 \\ 0 \\ 1
    \end{pmatrix} \nonumber\\
    & =\begin{pmatrix}
      1 \\ 1 \\ 1
    \end{pmatrix}\otimes\begin{pmatrix}
      5 \\ 6 \\ 7
    \end{pmatrix}\otimes\begin{pmatrix}
      1 \\ 1 \\ 1
    \end{pmatrix}+\begin{pmatrix}
      1 \\ 1 \\ 1
    \end{pmatrix}\otimes\begin{pmatrix}
      -1 \\ 0 \\ 1
    \end{pmatrix}\otimes\begin{pmatrix}
      1 \\ 1 \\ 1
    \end{pmatrix}+\begin{pmatrix}
      1 \\ 1 \\ 1
    \end{pmatrix}\otimes\begin{pmatrix}
      1 \\ 1 \\ 1
    \end{pmatrix}\otimes\begin{pmatrix}
      -1 \\ 0 \\ 1
    \end{pmatrix} \nonumber
  \end{align}
  So $M$ can be decomposed as three $rank$ 1 tensors, $\therefore 2\leq rank(M)\leq 3$
\end{solution}



\end{CJK}
\end{document}