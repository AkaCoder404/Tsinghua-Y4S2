\documentclass[10pt, a4paper, oneside]{article}
\usepackage{CJKutf8}
\usepackage{amsmath, amsthm, amssymb, bm, color, framed, graphicx, hyperref, mathrsfs}
\usepackage{fancyhdr}
\usepackage{multicol}
\usepackage[left=20mm, right=20mm, top=20mm, bottom=20mm]{geometry}
\pagestyle{fancy}
\fancyhf{}
\rhead{\begin{CJK}{UTF8}{gbsn}{计83 李天勤 2018080106}\end{CJK}}
\lhead{\begin{CJK}{UTF8}{gbsn}{高代线性代数作业3}\end{CJK}}
	

\title{\textbf{课程作业}}
\author{AkaCoder404}
\date{\today}
\newpage
\linespread{1.5}
\definecolor{shadecolor}{RGB}{241, 241, 255}
\newcounter{problemname}
\newenvironment{problem}{\begin{shaded}\stepcounter{problemname}\par\noindent\textbf{题目\arabic{problemname}. }}{\end{shaded}\par}
\newenvironment{solution}{\par\noindent\textbf{解答. }}{\par}
\newenvironment{note}{\par\noindent\textbf{题目\arabic{problemname}的注记. }}{\par}

\begin{document}
\begin{CJK}{UTF8}{gbsn}

% \maketitle
% \newpage

\section{1.3 Jordan Canonical Form}

\setcounter{problemname}{0}
\begin{problem}
  Find a basis in the following vector space so that the linear map involved will be in Jordan normal form. Also find the Jordan normal form.
\end{problem}

\begin{solution} \newline
  $(1)$ $\because V$ is a real vector space, $V \subset \mathbb{R}^4$ \newline
  Let $x = a + bi, y = c + di$ \newline
  $$ V \begin{bmatrix}
    a \\ b \\ c \\ d
  \end{bmatrix} = 
  \begin{bmatrix}
    a - c \\ -b \\ b - c \\ b-d
  \end{bmatrix}, V = \begin{bmatrix}
    -1 & 0 & -1 & 0 \\
    0 & -1 & 0 & 0 \\
    0 &  1 & -1 & 0 \\
    0 & 1 & 0 & -1
  \end{bmatrix}$$ Get Eigenvalues,
  $$ \text{det}(XI-V) = \left| \begin{matrix}
    x-1 && 1 \\
    & x + 1 && \\
    & -1 & x + 1 & \\
    & -1 && x + 1
  \end{matrix}\right|  = (x + 1)^3(x-1)$$ 
  \begin{table}[h!]
    \begin{center}
      % \caption{Your first table.}
      \label{tab:table1}
      \begin{tabular}{l|c} % <-- Alignments: 1st column left, 2nd middle and 3rd right, with vertical lines in between
        % \textbf{Value 1} & \textbf{Value 2} & \textbf{Value 3}\\
        eigen & alg mult \\
        \hline
        1 & 1  \\
        -1 & 3 \\
      \end{tabular}
    \end{center}
  \end{table} \newline
  $\text{Ker}(V-I)$ = $\text{span}\left\{ \begin{pmatrix} (1 \\ 0 \\ 0 \\ 0)\end{pmatrix}\right\}$, $V_1 = \begin{pmatrix}
    1 \\ 0 \\ 0 \\ 0
  \end{pmatrix}$ \newline
  $\text{Ker}(V+I)^2$ = $\text{span}\left\{ \begin{pmatrix} 1 \\ 4 \\ 0 \\ 0, \end{pmatrix}, \begin{pmatrix} 1 \\ 0 \\ 2 \\ 0 \end{pmatrix}, \begin{pmatrix} 0 \\ 0 \\ 0 \\ 1 \end{pmatrix}  \right\}$ \newline
  $\text{Ker}(V+I)$ = $\text{span}\left\{ \begin{pmatrix} 1 \\ 0 \\ 2 \\ 0 \end{pmatrix}, \begin{pmatrix} 0 \\ 0 \\ 0 \\ 1 \end{pmatrix}  \right\}$ \newline
  $V_2 = \begin{pmatrix}
    1 \\ 4 \\ 0 \\ 0
  \end{pmatrix}, (V+I)V_2 = \begin{pmatrix}
    2 \\ 0 \\ 4 \\ 4
  \end{pmatrix}, V_3 = \begin{pmatrix}
    0 \\ 0 \\ 1 \\ 0
  \end{pmatrix}$
  $$ \therefore V = \begin{pmatrix}
    1 & 2 & 1 & 0 \\
    0 & 0 & 4 & 0 \\ 
    0 & 4 & 0 & 1 \\
    0 & 4 & 0 & 0
  \end{pmatrix}\begin{pmatrix}
    1 \\
    & -1 & 1 \\
    && -1 \\
    &&&-1 
  \end{pmatrix}\begin{pmatrix}
    1 & 2 & 1 & 0 \\
    0 & 0 & 4 & 0 \\ 
    0 & 4 & 0 & 1 \\
    0 & 4 & 0 & 0
  \end{pmatrix}^{-1}$$ \newline
  $(2)$ 
  $$V\begin{pmatrix}
    a \\ b \\ c \\ d \\ e 
  \end{pmatrix} = \begin{pmatrix}
    0 \\ 4a \\ 3b + d \\ 2c \\  d + e
  \end{pmatrix}, V = \begin{pmatrix}
    0 & 0 & 0 & 0 & 0 \\
    4 & 0 & 0 & 0 & 0 \\ 
    0 & 3 & 0 & 1 & 0 \\
    0 & 0 & 2 & 0 & 0 \\
    0 & 0 & 0 & 1 & 1
  \end{pmatrix}$$
  $$ \text{det}(XI-V) = \left| \begin{matrix}
    X & 0 & 0 & 0 & 0 \\
    -4 & X & 0 & 0 & 0 \\
    0 & -3 & X & -1 & 0 \\
    0  & 0 & -2 & X & 0 \\
    0 & 0 & 0 & -1 & X-1
  \end{matrix}\right| = (x-1)x^2(x-\sqrt{2}) (x+\sqrt{2}) $$ 
  \begin{table}[h!]
    \begin{center}
      % \caption{Your first table.}
      \label{tab:table2}
      \begin{tabular}{l|c} % <-- Alignments: 1st column left, 2nd middle and 3rd right, with vertical lines in between
        % \textbf{Value 1} & \textbf{Value 2} & \textbf{Value 3}\\
        eigen & alg mult \\
        \hline
        1 & 1  \\
        $\sqrt{2}$ & 1 \\
        $-\sqrt{2}$ & 1 \\ 
        0 & 2
      \end{tabular}
    \end{center}
  \end{table} \newline \newline
  $\text{Ker}(V-I) = \text{span}\left\{ \begin{pmatrix} 0 \\ 0 \\ 0 \\ 0 \\ 1 \end{pmatrix}\right\}, V_1 = \begin{pmatrix}
    0 \\ 0 \\ 0 \\ 0 \\ 1
  \end{pmatrix}$ \newline
  $\text{Ker}(V-\sqrt{2}I) = \text{span}\left\{ \begin{pmatrix} 0 \\ 0 \\ 1 - \sqrt{2} \\ \sqrt{2} - 1 \\ 1 \end{pmatrix}\right\}, V_2 = \begin{pmatrix}
    0 \\\ 0 \\ 1 - \frac{\sqrt{2}}{2} \\ \sqrt{2} - 1 \\ 1
  \end{pmatrix}$ \newline
  $\text{Ker}(V+\sqrt{2}I) = \text{span}\left\{ \begin{pmatrix} 0 \\ 0 \\ 1 + \sqrt{2} \\ -1 - \sqrt{2} \\ 1 \end{pmatrix}\right\}, V_3 = \begin{pmatrix}
    0 \\\ 0 \\ 1 + \frac{\sqrt{2}}{2} \\ -1 - \sqrt{2} \\ 1
  \end{pmatrix}$ \newline
  $\text{Ker}(V) = \text{span}\left\{\begin{pmatrix} 0 \\ 1 \\ 0 \\ -3 \\ 3 \end{pmatrix}\right\}, \text{Ker}(V^2) = \text{span}\left\{  \begin{pmatrix}
      1 \\ -4 \\ -6 \\ 0 \\ 12
    \end{pmatrix}, \begin{pmatrix}
      1 \\ 0 \\ -6 \\ 0 \\ 12
    \end{pmatrix}
  \right\}$ \newline
  $V_4 = \begin{pmatrix}
    1 \\ 0 \\ -6 \\ 0 \\ 12
  \end{pmatrix}, VV_4 = \begin{pmatrix}
    0 \\ 4 \\ 0 \\ -12 \\ 12
  \end{pmatrix}$
  $$ \therefore V = \underbrace{\begin{pmatrix} 
    0 & 0 & 0 & 0 & 1 \\
    0 & 0 & 0 & 4 & 0 \\
    0 & 1 - \frac{\sqrt{2}}{2} & 1 + \frac{\sqrt{2}}{2} & 0 & -6 \\
    0 & \sqrt{2} - 1 &-1 - \sqrt{2}& -12 & 0 \\ 
    1 & 1 & 1 & 12 & 12
  \end{pmatrix}}_{X}\begin{pmatrix}
    1 \\
    &\sqrt{2} \\ 
    &&-\sqrt{2} \\
    &&&0 & 1\\
    &&&& 0
  \end{pmatrix}\underbrace{\begin{pmatrix} 
    0 & 0 & 0 & 0 & 1 \\
    0 & 0 & 0 & 4 & 0 \\
    0 & 1 - \frac{\sqrt{2}}{2} & 1 + \frac{\sqrt{2}}{2} & 0 & -6 \\
    0 & \sqrt{2} - 1 &-1 - \sqrt{2}& -12 & 0 \\ 
    1 & 1 & 1 & 12 & 12 
  \end{pmatrix}^{-1}}_{X^{-1}}$$ \newline
  $(3)$ $\det(XI-A) = (x^2-a_1a_4)(x^2-a_2a_3)$ \newline
  $ \text{Let } \sqrt{x} = 
    \begin{cases}
    \sqrt{x} &  x\geq 0 \\
    -i\sqrt{x} & x < 0 
    \end{cases}$ \newline
  Case 1: $a_1 a_4 \neq a_2a_3$ and $a_1a_2 \neq 0, a_2a_3\neq 0$ \newline 
  Eigenvalues are $\sqrt{a_1a_4}, -\sqrt{a_1a_4}, \sqrt{a_2a_3}, -\sqrt{a_2a_3}$ 
  $$ A = \begin{pmatrix}
    \sqrt{\frac{a_1}{a_4}} & -  \sqrt{\frac{a_1}{a_4}} & 0 & 0  \\ 
    0 & 0 &   \sqrt{\frac{a_2}{a_3}} & -   \sqrt{\frac{a_2}{a_3}}\\
    0 & 0 & 1 & 1 \\ 
    1 & 1 & 0 & 0
  \end{pmatrix} \begin{pmatrix}
    \sqrt{a_1a_4} \\ & -\sqrt{a_1a_4} \\ && \sqrt{a_2a_3} \\ &&& -\sqrt{a_2a_3}
  \end{pmatrix} \begin{pmatrix}
    \sqrt{\frac{a_1}{a_4}} & -  \sqrt{\frac{a_1}{a_4}} & 0 & 0  \\ 
    0 & 0 &   \sqrt{\frac{a_2}{a_3}} & -   \sqrt{\frac{a_2}{a_3}}\\
    0 & 0 & 1 & 1 \\ 
    1 & 1 & 0 & 0
  \end{pmatrix}^{-1}$$
  Case 2: $a_1a_4 = a_2a_3 \neq 0$ \newline 
  Eigenvalues are $\sqrt{a_1a_4}, -\sqrt{a_1a_4}$ 
  $$ A = \begin{pmatrix}
    \sqrt{\frac{a_1}{a_4}} & 0 & -  \sqrt{\frac{a_1}{a_4}} & 0 \\
    0 & \sqrt{\frac{a_2}{a_3}} & 0 & -\sqrt{\frac{a_2}{a_3}} \\ 
    0 & 1 & 0 & 1 \\ 
    1 & 0 & 1 & 0
  \end{pmatrix} \begin{pmatrix}
    \sqrt{a_1a_4} \\ & \sqrt{a_1a_4}  \\ && -\sqrt{a_1a_4}  \\ &&& -\sqrt{a_1a_4} 
  \end{pmatrix} \begin{pmatrix}
    \sqrt{\frac{a_1}{a_4}} & 0 & -  \sqrt{\frac{a_1}{a_4}} & 0 \\
    0 & \sqrt{\frac{a_2}{a_3}} & 0 & -\sqrt{\frac{a_2}{a_3}} \\ 
    0 & 1 & 0 & 1 \\ 
    1 & 0 & 1 & 0
  \end{pmatrix}^{-1}$$ 
  Case 3: $a_1a_4=0, \bar{V}a_2a_3 = 0, a_1a_4\neq a_2a_3$ \newline
  Eigenvalues are $0(2), \sqrt{a_2a_3}, -\sqrt{a_2a_3}$ \newline
  1. If $a_1=a_4 = 0$ 
  $$ A = \begin{pmatrix}
    1 & 0 & 0 & 0 \\
    0 & 0 & \sqrt{\frac{a2}{a3}} & -\sqrt{\frac{a2}{a3}} \\
    0 & 0 & 1 & 1 \\
    0 & 1 & 0 & 0
  \end{pmatrix} \begin{pmatrix}
    0 \\ &0 \\ && \sqrt{a_2a_3} \\ &&& -\sqrt{a_2a_3} \\
  \end{pmatrix} \begin{pmatrix}
    1 & 0 & 0 & 0 \\
    0 & 0 & \sqrt{\frac{a2}{a3}} & -\sqrt{\frac{a2}{a3}} \\
    0 & 0 & 1 & 1 \\
    0 & 1 & 0 & 0
  \end{pmatrix}^{-1}$$
  2. If $a_1=0,a_4\neq 0$ 
  $$ A = \begin{pmatrix}
    0 & 1 & 0 & 0 \\
    0 & 0 & \sqrt{\frac{a2}{a3}} & -\sqrt{\frac{a2}{a3}} \\
    0 & 0 & 1 & 1 \\
    a_4 & 0 & 0 & 0
  \end{pmatrix} \begin{pmatrix}
    0 & 1 \\ &0 \\ && \sqrt{a_2a_3} \\ &&& -\sqrt{a_2a_3} \\
  \end{pmatrix} \begin{pmatrix}
    0 & 1 & 0 & 0 \\
    0 & 0 & \sqrt{\frac{a2}{a3}} & -\sqrt{\frac{a2}{a3}} \\
    0 & 0 & 1 & 1 \\
    a_4 & 0 & 0 & 0
  \end{pmatrix}^{-1}$$
  Case 4: $a_1a_4 = a_2a_3 = 0$ \newline
  1. If 4 of them are zero, $a_1 = a_2 = a_3 = a_4 = 0$ 
  $$A = I0I^{-1}$$
  2. If 3 of them are zero, $a_1 = a_2 = a_3 = 0, a_4 \neq 0 $ \newline
  $$A = \begin{pmatrix}
    0 & 1 & 0 & 0 \\ 
    0 & 0 & 1 & 0 \\
    0 & 0 & 0 & 1 \\
    a^4 & 0 & 0 & 0
  \end{pmatrix} \begin{pmatrix}
    0 & 1 \\ & 0 \\ && 0 \\ &&& 0  
  \end{pmatrix} \begin{pmatrix}
    0 & 1 & 0 & 0 \\ 
    0 & 0 & 1 & 0 \\
    0 & 0 & 0 & 1 \\
    a^4 & 0 & 0 & 0
  \end{pmatrix}^{-1}$$
  3. If 2 of them are zero, $a_1 = a_2 = 0, a_3, a_4 \neq 0 $ \newline
  $$A = \begin{pmatrix}
    0 & 1 & 0 & 0 \\ 
    0 & 0 & 0 & 1 \\
    0 & 0 & a_3 & 0 \\
    a^4 & 0 & 0 & 0
  \end{pmatrix} \begin{pmatrix}
    0 & 1 \\ & 0 \\ && 0 & 1 \\ &&& 0  
  \end{pmatrix} \begin{pmatrix}
    0 & 1 & 0 & 0 \\ 
    0 & 0 & 0 & 1 \\
    0 & 0 & a_3 & 0 \\
    a^4 & 0 & 0 & 0
  \end{pmatrix}^{-1}$$

\end{solution}


\begin{problem}
 A partition of integer $n$ is a way to write $n$ as a sum of other positive integers.
\end{problem}

\begin{solution} \newline
  $(1)$ For the partition $n = a_1 + \cdots + a_k$. \newline 
    If we look at each column, column $a_i$ represents the length of a chain \newline
    If we look at each row, row $j$ represents how many eigen vectors that we can find in $\text{Ker}(A^j)$, but not in $\text{Ker}(A^{j-1})$  \newline 
    Thus, it is equal to $\dim{\ker(A^j)} - \dim{\ker(A^{j-1})}$, therefore, the two dot diagrams are transpose of each other. \newline
  $(2)$ The total number of dots that are in the first column and the first row is $2a_1-1$, which is an odd number \newline
  $\forall$ self-conjugat partitions the partition of distinct odd numbers is $n = 2a_1 - 1 + 2a_2-3 + \cdots + 2a_k - (k+1), k = n$, and 
  $\forall$ distinct odd number partition n = $b_1 + \cdots + b_k$, where $a_1 = \frac{b_1 + 1}{2}, \dots, a_k = \frac{b_k + 2k-1}{2}$ \newline
  $(3)$ Because $A$ is upper triangular and nilpotent \newline
  $A$ takes the form $\begin{pmatrix}
    0 & * & * & * \\
    0 & 0 & * & * \\
    0 & 0 & 0 & * \\
    0 & 0 & 0 & 0 
  \end{pmatrix}$ if none of $a_ij$ is zero, and that $A$ only has one eigenvalue of 0 \newline
  $$ \therefore \dim{A} = 3, \dim\ker{A} = 1$$
  $A^2$ takes the form $\begin{pmatrix}
    0 & 0 & * & * \\
    0 & 0 & 0 & * \\
    0 & 0 & 0 & 0 \\
    0 & 0 & 0 & 0 
  \end{pmatrix} \Rightarrow \dim{A^2} = \dim\ker{A^2} = 2$ \newline
  $A^3$ takes the form $\begin{pmatrix}
    0 & 0 & 0 & * \\
    0 & 0 & 0 & 0 \\
    0 & 0 & 0 & 0 \\
    0 & 0 & 0 & 0 
  \end{pmatrix} \Rightarrow \dim{A^3} = 1 , \dim\ker{A^3} = 3$ \newline 
  $A^4 = 0, \dim\ker{A^4} = 4 \Rightarrow$ the chain of $A$ is $\begin{pmatrix}
    0 \\ 0 \\ 0 \\  1
  \end{pmatrix}, A\begin{pmatrix}
    0 \\ 0 \\ 0 \\ 1 
  \end{pmatrix}, A^2\begin{pmatrix}
    0 \\ 0 \\ 0 \\ 1 
  \end{pmatrix}, A^3\begin{pmatrix}
    0 \\ 0 \\ 0 \\ 1 
  \end{pmatrix}$ \newline
  Thus $A$'s Jordan canonical form has only one Jordan block, and the possibiliy is $100\%$

\end{solution}

\vfill
\vspace{0.25\textheight}
\newpage





\end{CJK}
\end{document}