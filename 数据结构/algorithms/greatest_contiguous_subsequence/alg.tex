\documentclass[10pt]{article}
\usepackage{algorithm}
\usepackage{algpseudocode}
\usepackage{fullwidth}
\algrenewcommand\textproc{}
\algrenewcommand\algorithmicrequire{\textbf{Input:}}
\algrenewcommand\algorithmicensure{\textbf{Output:}}


\errorcontextlines\maxdimen

% begin vertical rule patch for algorithmicx 
\makeatletter
% start with some helper code
% This is the vertical rule that is inserted
\newcommand*{\algrule}[1][\algorithmicindent]{\makebox[#1][l]{\hspace*{.5em}\thealgruleextra\vrule height \thealgruleheight depth \thealgruledepth}}%
% its height and depth need to be adjustable
\newcommand*{\thealgruleextra}{}
\newcommand*{\thealgruleheight}{.75\baselineskip}
\newcommand*{\thealgruledepth}{.25\baselineskip}

\newcount\ALG@printindent@tempcnta
\def\ALG@printindent{%
    \ifnum \theALG@nested>0% is there anything to print
        \ifx\ALG@text\ALG@x@notext% is this an end group without any text?
            % do nothing
        \else
            \unskip
            \addvspace{-1pt}% FUDGE to make the rules line up
            % draw a rule for each indent level
            \ALG@printindent@tempcnta=1
            \loop
                \algrule[\csname ALG@ind@\the\ALG@printindent@tempcnta\endcsname]%
                \advance \ALG@printindent@tempcnta 1
            \ifnum \ALG@printindent@tempcnta<\numexpr\theALG@nested+1\relax% can't do <=, so add one to RHS and use < instead
            \repeat
        \fi
    \fi
    }%
\usepackage{etoolbox}
% the following line injects our new indent handling code in place of the default spacing
\patchcmd{\ALG@doentity}{\noindent\hskip\ALG@tlm}{\ALG@printindent}{}{\errmessage{failed to patch}}
\makeatother

% the required height and depth are set by measuring the content to be shown
% this means that the content is processed twice
\newbox\statebox
\newcommand{\myState}[1]{%
    \setbox\statebox=\vbox{#1}%
    \edef\thealgruleheight{\dimexpr \the\ht\statebox+1pt\relax}%
    \edef\thealgruledepth{\dimexpr \the\dp\statebox+1pt\relax}%
    \ifdim\thealgruleheight<.75\baselineskip
        \def\thealgruleheight{\dimexpr .75\baselineskip+1pt\relax}%
    \fi
    \ifdim\thealgruledepth<.25\baselineskip
        \def\thealgruledepth{\dimexpr .25\baselineskip+1pt\relax}%
    \fi
    %\showboxdepth=100
    %\showboxbreadth=100
    %\showbox\statebox
    \State #1%
    %\State \usebox\statebox
    %\State \unvbox\statebox
    %reset in case the next command is not wrapped in \myState
    \def\thealgruleheight{\dimexpr .75\baselineskip+1pt\relax}%
    \def\thealgruledepth{\dimexpr .25\baselineskip+1pt\relax}%
}
% end vertical rule patch for algorithmicx



\begin{document}
% control width of algorithm boxes
\begin{fullwidth}[width=\linewidth+4cm,leftmargin=-2cm,rightmargin=-2cm] 



\begin{algorithm}[H]
  \caption{Brute Force Method}
  \begin{algorithmic}[1]
    \Require $a$ - array of integers
    \Require $n$ - size of array
    \Ensure $sum$ - is the maximum sum of a contiguous subarray in $a$
    \Procedure{GS}{$a$}
      \State MaxSum $\gets$ a[0] \Comment{current maximum sum}
      \For{i = 0 to n - 1}                   
        \For{j = i to n - 1}                  
        \State CurrentSum $\gets$ 0    \Comment{sum of subarray}
          \For{k = i to j}                     
          \State CurrentSum $\gets$ CurrentSum + a[k] 
          \EndFor
          \If{MaxSum $<$ CurrentSum} 
            \State MaxSum $\gets$ CurrentSum \Comment {update maximum sum}
          \EndIf
        \EndFor
      \EndFor
    \EndProcedure
  \end{algorithmic}
\end{algorithm}

\begin{algorithm}[H]
  \caption{Sliding Window Algorithm}
  \begin{algorithmic}[1]
    \Require $a$ - array of integers
    \Require $n$ - size of array
    \Ensure $sum$ - is the maximum sum of a contiguous subarray in $a$
    \Procedure{GS}{$a$}
      \State MaxSum $\gets$ a[0] \Comment{current maximum sum}
      \For{i = 0 to n - 1}                
        \State CurrentSum $\gets$ 0
        \For{j = i to n - 1}   
          \State CurrentSum $\gets$ CurrentSum + $a[j]$;
        \If{MaxSum $<$ CurrentSum}  \Comment{update maximumx sum}
          \State MaxSum $\gets$ CurrentSum
        \EndIf
        \EndFor
      \EndFor
    \EndProcedure
  \end{algorithmic}
\end{algorithm}


\begin{algorithm}[H]
  \caption{Kadane's Algorithm}
  \begin{algorithmic}[1]
    \Require $a$ - array of integers
    \Require $n$ - size of array
    \Ensure $sum$ - is the maximum sum of a contiguous subarray in $a$
    \Procedure{GS}{$a$}
      \State MaxSum $\gets$ a[0] \Comment{current maximum sum}
      \State CurrentSum $\gets$   MaxSum
      \For{i = 0 to n - 1}                
        \State CurrentSum $\gets$ CurrentSum + $a[i]$   
        \If{CurrentSum $<$ a[i]}  \Comment{current sum up to this less than current point}
          \State CurrentSum $\gets a[i]$
        \EndIf
        \If{MaxSum $<$ CurrentSum}  \Comment{update maximumx sum}
          \State MaxSum $\gets$ CurrentSum
        \EndIf
      \EndFor
    \EndProcedure
  \end{algorithmic}
\end{algorithm}




\begin{algorithm}[H]
  \caption{Brute Force Method}
  \begin{algorithmic}[1]
    \Require $x$ - decision tree
    \Ensure $abc$ is $x$ in Txx Bxxxxx Gxxxx
  \Procedure{Recursion}{$a$}
      \State $a\gets$ \Call{Recursion}{$a$} \Comment{Call Recursion again}
      \State \textbf{return} $a$ 
  \EndProcedure
  \end{algorithmic}
\end{algorithm}



\begin{algorithm}[H]
  \caption{Euclid’s algorithm}\label{euclid}
  \begin{algorithmic}[1]
    \Procedure{Euclid}{$a,b$}\Comment{The g.c.d.\ of a and b}
      \myState{$r\gets a\bmod b$}
      \While{$r\not=0$}\Comment{We have the answer if r is 0}
        \myState{$a\gets\displaystyle\sum_{i=1}^n x_i$}\Comment{Nonsense to show that tall lines might work}
        \myState{$a\gets b$}
        \myState{$b\gets r$}
        \myState{$r\gets a\bmod b$}
      \EndWhile\label{euclidendwhile}
      \myState{\Return $b$\Comment{The gcd is b}}
    \EndProcedure
  \end{algorithmic}
\end{algorithm}


\end{fullwidth}
\end{document}




% \begin{document}

% \begin{algorithm}
%   \caption{Euclid’s algorithm}\label{euclid}
%   \begin{algorithmic}[1]
%     \Procedure{Euclid}{$a,b$}\Comment{The g.c.d.\ of a and b}
%       \myState{$r\gets a\bmod b$}
%       \While{$r\not=0$}\Comment{We have the answer if r is 0}
%         \myState{$a\gets\displaystyle\sum_{i=1}^n x_i$}\Comment{Nonsense to show that tall lines might work}
%         \myState{$a\gets b$}
%         \myState{$b\gets r$}
%         \myState{$r\gets a\bmod b$}
%       \EndWhile\label{euclidendwhile}
%       \myState{\Return $b$\Comment{The gcd is b}}
%     \EndProcedure
%   \end{algorithmic}
% \end{algorithm}
% \end{document}