\documentclass[10pt]{article}
\usepackage{algorithm}
\usepackage{algpseudocode}
\usepackage{fullwidth}
\algrenewcommand\textproc{}
\algrenewcommand\algorithmicrequire{\textbf{Input:}}
\algrenewcommand\algorithmicensure{\textbf{Output:}}


\errorcontextlines\maxdimen

% begin vertical rule patch for algorithmicx 
\makeatletter
% start with some helper code
% This is the vertical rule that is inserted
\newcommand*{\algrule}[1][\algorithmicindent]{\makebox[#1][l]{\hspace*{.5em}\thealgruleextra\vrule height \thealgruleheight depth \thealgruledepth}}%
% its height and depth need to be adjustable
\newcommand*{\thealgruleextra}{}
\newcommand*{\thealgruleheight}{.75\baselineskip}
\newcommand*{\thealgruledepth}{.25\baselineskip}

\newcount\ALG@printindent@tempcnta
\def\ALG@printindent{%
    \ifnum \theALG@nested>0% is there anything to print
        \ifx\ALG@text\ALG@x@notext% is this an end group without any text?
            % do nothing
        \else
            \unskip
            \addvspace{-1pt}% FUDGE to make the rules line up
            % draw a rule for each indent level
            \ALG@printindent@tempcnta=1
            \loop
                \algrule[\csname ALG@ind@\the\ALG@printindent@tempcnta\endcsname]%
                \advance \ALG@printindent@tempcnta 1
            \ifnum \ALG@printindent@tempcnta<\numexpr\theALG@nested+1\relax% can't do <=, so add one to RHS and use < instead
            \repeat
        \fi
    \fi
    }%
\usepackage{etoolbox}
% the following line injects our new indent handling code in place of the default spacing
\patchcmd{\ALG@doentity}{\noindent\hskip\ALG@tlm}{\ALG@printindent}{}{\errmessage{failed to patch}}
\makeatother

% the required height and depth are set by measuring the content to be shown
% this means that the content is processed twice
\newbox\statebox
\newcommand{\myState}[1]{%
    \setbox\statebox=\vbox{#1}%
    \edef\thealgruleheight{\dimexpr \the\ht\statebox+1pt\relax}%
    \edef\thealgruledepth{\dimexpr \the\dp\statebox+1pt\relax}%
    \ifdim\thealgruleheight<.75\baselineskip
        \def\thealgruleheight{\dimexpr .75\baselineskip+1pt\relax}%
    \fi
    \ifdim\thealgruledepth<.25\baselineskip
        \def\thealgruledepth{\dimexpr .25\baselineskip+1pt\relax}%
    \fi
    %\showboxdepth=100
    %\showboxbreadth=100
    %\showbox\statebox
    \State #1%
    %\State \usebox\statebox
    %\State \unvbox\statebox
    %reset in case the next command is not wrapped in \myState
    \def\thealgruleheight{\dimexpr .75\baselineskip+1pt\relax}%
    \def\thealgruledepth{\dimexpr .25\baselineskip+1pt\relax}%
}
% end vertical rule patch for algorithmicx



\begin{document}
% control width of algorithm boxes
\begin{fullwidth}[width=\linewidth+4cm,leftmargin=-2cm,rightmargin=-2cm] 



\begin{algorithm}[H]
  \caption{Karatsuba Algorithm}
  \begin{algorithmic}[1]
    \Require $a, b$ - two numbers represent as strings
    \Require $n$ - length of the two strings
    \Ensure $product$ - the product of two numbers
    \Procedure{Karatsuba}{$a$, $b$}
      \State PadZeroes(a, b)  \Comment{make sure strings are of equal length}
      \If{a and b are single digit numbers} \Comment{base case} 
        \State return a * b
      \EndIf
      \State $al \gets a[0,n/2)$    \Comment{left of a}
      \State $ar \gets a[n/2,n)$    \Comment{right of a}
      \State $bl \gets b[0,n/2)$    \Comment{left of b}
      \State $br \gets b[n/2,n)$    \Comment{right of b}
      \State $p1 \gets Karatsuba(al, bl)$
      \State $p2 \gets Karatsuba(al + ar, bl + br)$
      \State $p3 \gets Karatsuba(al, bl)$ 
      \State return $(10^n * p1) + 10^{n/2}*(p2 - p1 - p3) + p3$
  
    \EndProcedure
  \end{algorithmic}
\end{algorithm}


\begin{algorithm}[H]
  \caption{Graphics}
  \begin{algorithmic}[1]
    \Require $x, y$ - arrays representing points on x and y axis
    \Require $n$ - the number of $x$ and $y$ points given
    \Require $p$ - point given
    \Ensure how many lines OP intersects
    \Procedure{Graphics}{$x$, $y$}
      \State QuickSort($x$)
      \State QuickSort($y$)
      \For{i = 0 to n - 1}
        \If{$(0 - x[i]) * (p.y - 0) - (y[i] - 0) * (p.x - x[i]) > 0$} \Comment{ToLeft algorithm}
          \State return $i$
        \EndIf
      \EndFor
      \State return n - 1; \Comment{points intersect all numbers}
    \EndProcedure
  \end{algorithmic}
\end{algorithm}

\begin{algorithm}[H]
  \caption{Long Multiplication}
  \begin{algorithmic}[1]
    \Require $a,b$ - two numbers represented as strings
    \Require $m,n$ - the length of two strings
    \Ensure the product of the two numbers
    \Procedure{LongMult}{$a$, $b$}
      \State Prod $\gets$ array[m + n] \Comment{result is at most size m + n} 
      \State Carry $\gets$ 0 \Comment{keep track of carry}
      \If{$n < m$} Swap($a,b$)  \Comment{smaller number on bottom}
      \EndIf
      \For{i = m - 1 to 0}  \Comment{multiply starting from end of smaller number}
        \For{j = n - 1 to 0}  \Comment{multiply digit by digit of larger number}  
          \State TempProd $\gets$ Prod[i + j + 1] + (a[i] * a[j])  \Comment{store ones digit}
          \State Prod[i + j + 1] $\gets$ TempSum mod $10$;
          \State Prod[i + j]  $\gets$  TempSum / 10  \Comment{store sum carry in previous element}
        \EndFor
      \EndFor
      \State return RemoveLeadingZero(Sum) \Comment{Remove leading zero}
    \EndProcedure
  \end{algorithmic}
\end{algorithm}



\end{fullwidth}
\end{document}



