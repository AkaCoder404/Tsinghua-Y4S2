\documentclass[12pt, a4paper, oneside]{article}
\usepackage{CJKutf8}
\usepackage{amsmath, amsthm, amssymb, bm, color, framed, graphicx, hyperref, mathrsfs}
\usepackage{fancyhdr}
\usepackage[left=20mm, right=20mm, top=20mm, bottom=20mm]{geometry}
\pagestyle{fancy}
\fancyhf{}
\setlength{\headheight}{14.60191pt}
\rhead{\begin{CJK}{UTF8}{gbsn}{计83 李天勤 2018080106}\end{CJK}}
\lhead{\begin{CJK}{UTF8}{gbsn}{初等数论及应用作业1}\end{CJK}}
	

\title{\textbf{课程作业}}
\author{Dylaaan}
\date{\today}
\newpage
\linespread{1.5}
\definecolor{shadecolor}{RGB}{241, 241, 255}
\newcounter{problemname}
\newenvironment{problem}{\begin{shaded}\stepcounter{problemname}\par\noindent\textbf{题目\arabic{problemname}. }}{\end{shaded}\par}
\newenvironment{solution}{\par\noindent\textbf{解答. }}{\par}
\newenvironment{note}{\par\noindent\textbf{题目\arabic{problemname}的注记. }}{\par}
\newcommand{\TAG}[1]{\tag*{$\left\tagldelim\parbox{\tagwidth}{#1}\right\tagrdelim$}}

\begin{document}
\begin{CJK}{UTF8}{gbsn}

% \maketitle
% \newpage

\section{习题 1.1 整除性}

\begin{problem}
  设$n$是奇数,则$8 \mid  n^2 - 1$
\end{problem}

\begin{solution}
  If $n$ is an odd number, then $n=2k-1$, where $k,k\in\mathbb{N}$.
  Therefore $n^2-1=(2k-1)^2-1=4k^2-4k+1-1$ 
  $$=4k^2-4k$$
  $$=4k(k-1)$$
  For any $k > 1, k(k-1)$ will always be even, since an odd number multiplied by even number will always be even.
  Hence $k(k-1)$ will be divisible by 2, $2\mid k(k-1)$. Therefore, $4k(k-1)$ has 4 as a factor, and we know 
  $$ a\mid b \Rightarrow ac\mid bc $$.
  Thus, it can be divisible by $2\times4=8$ as well.  
\end{solution}

% \begin{note}
%     这里是注记. 
% \end{note}

\begin{problem}
  设$n\ge3$是奇数,证明:
  \begin{equation}
    \left( 1 + \frac{1}{2} + \frac{1}{3} + \dots + \frac{1}{n-1} \right) (n-1)! \tag{*}
  \end{equation}
  被$n$整除
\end{problem}

\begin{solution}
  If $n$ is an odd number, then $n=2k+1$, where $k,k\in\mathbb{N}$. We can rewrite the original equation
  $$ \left( 1 + \frac{1}{2} + \frac{1}{3} + \dots + \frac{1}{2k} \right) (2k)! $$ 
  \begin{equation}
    (2k)! + \frac{1}{2}(2k)! + \frac{1}{3}(2k)! + \dots + \frac{1}{2k}(2k)! \tag{**}
  \end{equation}
    And if we add the first and last term, second and second last, etc 
  \begin{equation}
   (2k)! + \frac{1}{2k}(2k)! =  (2k)!\frac{2k+1}{(2k)}
  \end{equation}
  \begin{equation}
    \frac{1}{2}(2k)! +  \frac{1}{2k-1}(2k)! = (2k)!\frac{2k+1}{2(2k-1)}
  \end{equation}
  \begin{equation}
    \frac{1}{3}(2k)! +  \frac{1}{2k-2}(2k)! = (2k)!\frac{2k+1}{3(2k-2)}
  \end{equation}
  \begin{equation}
    \dots \notag
  \end{equation}
  \begin{equation}
    \frac{1}{k}(2k)! +  \frac{1}{k + 1}(2k)! = (2k)!\frac{2k+1}{k(k+1)}
    \end{equation}
    We know that $(1)(2)(3)\dots(k)$ all have $2k+1$ as a factor, then we know that $(1)+(2)+(3)+\dots(k)$ can also be factored
    Thus, we know that $2k+1\mid (**)$. Therefore, $n\mid (*)$ is true.



\end{solution}

\begin{problem}
  设$m$和$n$是正整数$m\ge3$,证明$2^m-1\nmid 2^n+1$.
\end{problem}

\begin{solution}
   Proof by contradiction. Let us assume that $2^m-1\mid 2^n+1$.
   That means $2^n -1 > 2^m + 1$, $m \geq 3$, $n \ge m$. \newline
   Thus there exists $a \in \mathbb{Z}$, where $n = m -a$. 
   $$ 2^n + 1 = 2^{m+a} + 1 = 2^m2^a + 1 = 2^m2^a+1+2^a-2^a = 2^a(2^m - 1) + 2^a + 1$$
   If it is divisible, then
   $$ \Rightarrow 2^a + 1 \mid 2^m - 1$$
   Let $m>a, m=a+x$ \newline
   And repeat $\dots$ \newline
   We find that  $2^x + 1 < 2^m -1$ and the left hand size is always smaller. Which means that there exists decimal,
   but from $2^a + \dots$ are all integers, which is a contradiction. Therefore $2^m-1\nmid 2^n+1$
\end{solution}



\begin{problem}
  设$q$是大于1的整数,证明:
\end{problem}
\begin{solution}
  1. If we use the division algorithm, then there exists integer $k$ such that 
  $$ n = q_1b + a_0,  0 \leq a_0 \leq b - 1, q_1  \ge b$$
  $$ q_1 = q_2b + a_1, 0 \leq a_1 \leq b - 1, q_2 \ge b$$ 
  $$ \dots $$ 
  $$ q_{k-1} = q_{k}b + a_{k-1}, 0 \leq a_{k-1} \leq b - 1, 0 < q_k \leq b - 1$$
  And thus since $a_i$ and $q_i$ are both uniquely determined, $q_k = a_k$, where $0 < a_k   \leq b- 1$
  $$ n = q_1b + a_0 = (q_2b+a_1)b + a_0 = q_2b^2 + a_1b + a_0$$ 
  $$ = (q_3b + a_2)b^2 + a_1b + a_0 = q_3b^3 + a_2b^2 + a_1b + a_0$$
  $$ = \dots$$
  $$ = a_kb^k + a_{k-1}b^{k-1}  + \dots + a_1b + a_0$$
  2. 
\end{solution}


\begin{problem}
  设$a_1,\dots ,a_n$,为实数$(n\ge2)$,证明:
\end{problem}

\begin{solution}
  Since we know that $\{a\} = a - [a]$ or $[a] = a - \{a\}$ or $a = [a] + \{a\}$
  $$ [a_1] + [a_2] + \dots + [a_n] = (a_1 + \dots + a_n) - (\{a_1\} + \dots + \{a_n\}) $$
  $$ [a_1 + \dots + a_n] = (a_1 + \dots + a_n) - \{a_1 + \dots + a_n\} $$
  $$ \{a_1\} + \dots + \{a_n\} \ge \{a_1 + \dots + a_n\}  $$
  $$ \Rightarrow [a_1] + [a_2] + \dots + [a_n] \le [a_1 + \dots + a_n] $$ 
  We also know that $\{a_1\} + \dots + \{a_n\} < n$ since each element is less than 1.
  And we know by definition that, we can prove the second half of the equation
  $$ \{a_1 + \dots + a_n\} < 1$$ 
\end{solution}


\setcounter{problemname}{10}
\begin{problem}
  设$n$是正整数$n\ge2$,如果$n$没有小于或等于$\sqrt{n}$素数因子,则$n$是素数
\end{problem}

\begin{solution}
  Every integer $n>1$ can be uniquely expressed as a product of primes.
  So, $n$ is composite, with least prime divisor $p_0$, then $n = mp_0$ with $m>1$ and 
  no prime divisor of m less than $p_0$. Therefore, $m \geq p_0$ and so $ n \geq p_0^2$.
  If $n$ must have at least one prime divisor when $\leq \sqrt{n}$ whenever $n$  is composite.
  Thus, if $n$ has no prime divisor $\leq \sqrt{n}$ and $n>1$, then $n$ must be prime.
\end{solution}


\begin{problem}
  对于每个整数$n\ge3$,$n$和$n!$,之间必有素数,由此证明素数有无限多
\end{problem}

\begin{solution}
  The base case or this problem would be when $n=3$, such that $3<x<3!$, where $x=5$
  Let $p$ be any prime number that divides $n!-1$ \newline
  Since $p | n!-1$, $p$ does not divide $n!$ And it shows that $p$ cannot be equal to or less than $n$
  since $n!=n\times (n-1)!$, if it does, $p$ could divide $n!$, therefore
  $$ p > n $$
  Also, $p$ divides $n! - 1$ so, $p$ is less than or equal to $n! - 1$, and thus
  $$ n < p < n!$$
  There is a prime between $n$ and $n!$
\end{solution}  


\section{习题1.2 最大归约和最小公倍数}
\setcounter{problemname}{0}
\begin{problem}
  设$n$是正整数,证明$\frac{21n+4}{14n+3}$是既约分数
\end{problem}

\begin{solution}
  An irreducible fraction is one such that the numerator and denominator are integers that have no common
  divisors other than 1. In other words, a fraction a/b is irreducible iff a and b are coprime, $GCD(a,b) = 1$
  Using Euclid's Algorithm, and the divisibility algorithm $a = qb + r, 0 \le r \le b$ 
  $$ GCD(21n+4, 14n+3) = 21n+4=1(14n+3) + 7n+1 $$
  $$ GCD(14n+3, 7n+1) = 2(7n+1) + 1 $$
  $$ GCD(7n+1, 1) = 1 $$
  Thus we can say that $\frac{21n+4}{14n+3}$ is an irreducible fraction.

\end{solution}

\begin{problem}
  设$m,n$为正整数,$m$为奇数,证明
  $$ \left( 2^m-1,2^n+1 \right) = 1 $$
\end{problem}

\begin{solution}
  Using the result we have obtained from question 3, we can start of with 
  $$ \left( a^m-1,a^n-1 \right) = a^{(m,n)} - 1 $$
  Therefore we can conclude that 
  $$ \left( a^m-1,a^{2n}-1 \right) = a^{(m,2n)} - 1 $$
  However, since $m$ is odd, by Euclid's lemma, we know that $(m,2n) = (m,n)$, if and only if $m$ is odd , therefore
  $$ (2^n-1,2^n+1)= (2^n - 1,2) \mid 2 $$ 
  And if $ 2^{(m,2n)} - 1$ is odd, it implies, $(2^{(m,2n)}-1,2^n+1)=1 $
  $$ \Rightarrow (2^m - 1, 2^n + 1) = 1$$
\end{solution}


\begin{problem}
  设$m,n,a$均为正整数,$a\ge2$,证明:
  $$ \left( a^m-1,a^n-1 \right) = a^{(m,n)} - 1 $$
\end{problem}

\begin{solution}
  \newline
  Method 1:
  Assume $a,b \in\mathbb{Z}$
  $$ 2^{ab}-1=(2^a)^b-1 = (2^a -1)((2^a)^{b-1}+\dots+2^a +1)  = (2^a-1)\sum_{i=0}^{b-1}{(2^a)^i}$$
  $$ \Rightarrow 2^a - 1 \mid 2^{ab}-1 $$
  So if we let value $d = (m,n)$, we can also rewrite it as
  $$ 2^{m}-1=(2^d)^{\frac{m}{d}}-1 = (2^d -1) \sum_{i=0}^{\frac{m}{d}-1}{(2^d)^i}$$ 
  $$ \Rightarrow 2^{d}-1 \mid 2^m-1 $$
  $$ \Rightarrow 2^{d}-1 \mid 2^n-1 $$
  And we get that
  $$ \Rightarrow 2^{(m,n)}-1 \mid (2^m-1, 2^n-1) $$
  There must exist $a,b$, such that $  (m,n) = am - bn  $ and Let $ M = (2^m-1, 2^n-1)  $, we get 
  $$ M \mid 2^m-1 \Rightarrow M \mid 2^{am}-1 $$
  $$ M \mid 2^n-1 \Rightarrow M \mid 2^{bn} -1 $$ 
  Then
  $$ \Rightarrow M \mid ((2^{am} -1) -(2^{bn} -1)) $$ 
  $$ M \mid 2^{bn}*(2^{am-bn} -1) $$
  Subsituting, $(m,n)= am - bn$ back into the equation,
  $$ \Rightarrow M \mid 2^{(m,n)} - 1 $$
  $$ \therefore (2^m-1,2^n-1)=2^{(m,n)}-1 $$ 
  This works for base 2, as well as for any $a \geq 2$.  
\end{solution}

\begin{solution}
  \newline
  Method 2 \newline 
  Set $d = (m,n)$, $sd = m,td = n$ . Then 
  $$ a^m - 1 = (a^d)^s - 1 $$  
  and like before
  $$ a^d - 1 \mid (a^d)^s - 1 $$ 
  This goes for $a^n - 1$ as well
  $$ a^d - 1 \mid (a^d)^t - 1 $$ 
  Therefore 
  $$ \Rightarrow a^d - 1 \mid  (a^m-1,a^n-1) $$
  Now, by the Bachet-Bezout Theorem, there are integers $x,y$ with $(m,n) = mx+ny = d $ \newline
  However, x and y must have opposite signs. They can't both be negative, or $d$ would be negative. 
  They both cannot be positive either or else $ d \geq n + m $ when the conditions given were $ d \leq m, d \leq n $.
  So if we assume $ x > 0, y \leq 0 $, we get $(m,n) = mx-ny = d $.
  Setting $ t = (a^m-1, a^n-1)  $, we get 
  $$ t \mid (a^{mx} - 1) $$ 
  $$ t \mid (a^{-ny} - 1)$$ 
  $$ \Rightarrow t \mid ((2^{mx} - 1 )- a^d(2^{-ny} -1)) = a^d-1$$ 
  And the assertion is established

\end{solution}

\begin{solution}
  \newline
  Method 3 \newline
  This can be mimicked by an subtractive Euclidean algorithm $(n,m) = (n-m,m)$. For example
  $$ (f_5,f_2)=(f_3,f_2)=(f_1,f_2)=(f_1,f_1)=(f_1,f_0)=f_1=f_{(5,2)} $$ 
  such as 
  $$ (5, 2)=(3, 2)=(1, 2)=(1, 1)=(1, 0)=1 $$
  because 
  $$ f_n:= a^n-1 = a^{n-m}(a^m-1)+a^{n-m}-1 $$ 
  $$ \Rightarrow f_n=f_{n-m} + kf_m, k \in Z$$
  By induction, $n + m$, thereom obviously true for $n=m$ or $n=0$  or $m=0$.
  So we may assume  $n>m>0$, and we know that  $(f_n,f_m)=(f_{n-m},f_m)  $ because of Euclid and the  
  Since  $(n-m)+m < n+m$,  induction yields 
  $$(f_{n-m},f_m)=f_{(n-m,m)}=f_{(n,m)} $$
  And if we apply it to above, we know that 
  $$ \left( a^m-1,a^n-1 \right) = a^{(m,n)} - 1 $$
  This is known as a strong divisibility sequence
\end{solution}

\end{CJK}
\end{document}