\documentclass[12pt, a4paper, oneside]{article}
\usepackage{CJKutf8}
\usepackage{amsmath, amsthm, amssymb, bm, color, framed, graphicx, hyperref, mathrsfs}
\usepackage{fancyhdr}
\usepackage{multicol}
\usepackage[left=20mm, right=20mm, top=20mm, bottom=20mm]{geometry}
\pagestyle{fancy}
\fancyhf{}
\rhead{\begin{CJK}{UTF8}{gbsn}{计83 李天勤 2018080106}\end{CJK}}
\lhead{\begin{CJK}{UTF8}{gbsn}{初等数论及应用作业4}\end{CJK}}
	

\title{\textbf{课程作业}}
\author{AkaCoder404}
\date{\today}
\newpage
\linespread{1.5}
\definecolor{shadecolor}{RGB}{241, 241, 255}
\newcounter{problemname}
\newenvironment{problem}{\begin{shaded}\stepcounter{problemname}\par\noindent\textbf{题目\arabic{problemname}. }}{\end{shaded}\par}
\newenvironment{solution}{\par\noindent\textbf{解答. }}{\par}
\newenvironment{note}{\par\noindent\textbf{题目\arabic{problemname}的注记. }}{\par}

\begin{document}
\begin{CJK}{UTF8}{gbsn}

% \maketitle
% \newpage

\section{2.1}

\setcounter{problemname}{2}
\begin{problem}
  解下列同余方程 \newline
  $(1)$ $8x\equiv5\pmod{23}$ \newline
  $(2)$ $2x\equiv7\pmod{37}$
\end{problem}

\begin{solution}
  
\end{solution}


\vfill
\vspace{0.25\textheight}

\begin{problem}
   对每个整数$n$证明:\newline
   $(1)$ $n^2\not\equiv2\pmod{3}$ \newline
   $(2)$ $n^2\equiv0, 1\pmod{4}$ \newline
   $(3)$ $n^3\equiv0, 1, -1\pmod{9}$ \newline
   $(4)$ $n^4\equiv0,1\pmod{16}$ 
\end{problem}

\begin{solution} 

\end{solution}

\vfill
\vspace{0.25\textheight}

\newpage

\begin{problem}
   设$a$为奇数,$n\geq1$,证明$a^{2^n}\equiv1\pmod{2^{n+2}}$
\end{problem}

\begin{solution}

\end{solution}

\vfill
\vspace{0.25\textheight}


\section{2.4}
\setcounter{problemname}{0}
\begin{problem}
  解下列同余方程 \newline
   $(1)$ $32x\equiv12\pmod{8}$ \newline
   $(2)$ $28x\equiv124\pmod{116}$ \newline
   $(3)$ $5x\equiv44\pmod{8}$
\end{problem}

\begin{solution}

\end{solution}

\vfill
\vspace{0.25\textheight}
\newpage


\begin{problem}
  解下列同余方程组
  \begin{align*}
    (1) &\left\{
    \begin{aligned}
      x\equiv1\pmod{3},\\
      x\equiv1\pmod{5},\\
      x\equiv1\pmod{7}
    \end{aligned}\right.
    &
    (2) \left\{\begin{aligned}
      x\equiv2\pmod{4},\\
      x\equiv3\pmod{5},\\
      x\equiv7\pmod{9}
    \end{aligned}\right. 
    % \\
    % &\text{, pentru legea I, și}
    % &\text{pentru legea a II-a}
    \end{align*}
\end{problem}

\begin{solution}
  
\end{solution}

\newpage



% \vfill
% \vspace{0.3\textheight}



% \vfill
% \vspace{0.3\textheight}

% \newpage


\begin{problem}
 用中国剩余定理解同余方程$37x\equiv31\pmod{77}$
\end{problem}

\begin{solution}

\end{solution}   

\vfill
\vspace{0.25\textheight}

\begin{problem} 
  求$2^{400}$被319除的余数
\end{problem}

\begin{solution}

\end{solution}

\vfill
\vspace{0.25\textheight}
\newpage

\begin{problem}
  设$m_1,m_2$是正整数$b_1,b_2\in\mathbb{Z}$. 证明:同余方程组
  \begin{align*}
    (1) &\left\{
    \begin{aligned}
      x\equiv1\pmod{3},\\
      x\equiv1\pmod{5},\\
      x\equiv1\pmod{7}
    \end{aligned}\right.
    &&&&&&&&&&&&&&&&&&&&&&&&&
    % \\
    % &\text{, pentru legea I, și}
    % &\text{pentru legea a II-a}
    \end{align*} \newline
    有整数解的充分必要条件是$(m_1,m_2)\mid b_1 - b_2$,并且在次条件成立时,解为模$[m_1,m_2]$的同余类。
\end{problem}

\begin{solution}
  
\end{solution}

\vfill
\vspace{0.25\textheight}




\end{CJK}
\end{document}