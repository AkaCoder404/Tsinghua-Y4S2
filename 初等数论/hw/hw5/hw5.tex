\documentclass[12pt, a4paper, oneside]{article}
\usepackage{CJKutf8}
\usepackage{amsmath, amsthm, amssymb, bm, color, framed, graphicx, hyperref, mathrsfs}
\usepackage{fancyhdr}
\usepackage{multicol}
\usepackage[left=20mm, right=20mm, top=20mm, bottom=20mm]{geometry}
\pagestyle{fancy}
\fancyhf{}
\rhead{\begin{CJK}{UTF8}{gbsn}{计83 李天勤 2018080106}\end{CJK}}
\lhead{\begin{CJK}{UTF8}{gbsn}{初等数论及应用作业5}\end{CJK}}
	

\title{\textbf{课程作业}}
\author{AkaCoder404}
\date{\today}
\newpage
\linespread{1.5}
\definecolor{shadecolor}{RGB}{241, 241, 255}
\newcounter{problemname}
\newenvironment{problem}{\begin{shaded}\stepcounter{problemname}\par\noindent\textbf{题目\arabic{problemname}. }}{\end{shaded}\par}
\newenvironment{solution}{\par\noindent\textbf{解答. }}{\par}
\newenvironment{note}{\par\noindent\textbf{题目\arabic{problemname}的注记. }}{\par}

\begin{document}
\begin{CJK}{UTF8}{gbsn}

% \maketitle
% \newpage

\section{3.1}

\setcounter{problemname}{1}
\begin{problem}
  $a$对模$m$和$n$的阶分别为$s$和$t$,证明:$a$对模$[m,n]$的阶为$[s,t]$
\end{problem}

\begin{solution}
  \begin{equation}
    \because a^s \equiv \pmod{m}, a^t\equiv\pmod{n} 
  \end{equation}
  \begin{equation}
    \therefore a^{[s,t]} \equiv 1\pmod{[m,n]} 
  \end{equation}
  then using congruence property
  \begin{equation}
    a^{[s,t]}\equiv1\pmod{[m,n]}
  \end{equation}
  let there be an positive integer $k$, where $a^k\equiv1\pmod{mod[m,n]}$,then it is $a^k\equiv1\pmod{m}$
  $$ \because m \mid [m,n], [m,n] \mid (a,k - 1)$$
  $$ \therefore a^k \equiv 1\pmod{n}$$
  $$ \therefore s\mid k, t\mid k$$
  $\therefore k$ is $s,t$'s common multiple,thus $[s,t]\mid k$
  $$\therefore a \text{对模} [m,n] \text{的阶为} [s,t] $$ 

\end{solution}

\setcounter{problemname}{4}
\begin{problem}
  若$n$和$a$均是正整数,$a\geq2$, 证明:$n\mid \phi(a^n-1)$
\end{problem}

\begin{solution} 
  Let $m=a^n-1$ \newline
  Consider the group $G=(Z/mZ)^*$ or $(Z_m)^*$, which has $\phi(m)$ elements, the order of group is  $\phi(m)$ \newline
  Let $\bar{a} \in Z/mZ  $ be the remainder class of the integer $a$ modulo $m$, 
  $$ \because gcd(a,m)=gcd(a,a^n-1)=1$$
  $$ \therefore a\in G $$
  Consider the subgroup $H=\langle\bar{a}\rangle$ that is the subgroup generated by $\bar{a}$ \newline
  Now $a^n\equiv1\pmod{m} $ (where $m=a^n-1$ and $n$ is the smallest integer with this property) \newline
  but no positive integer $i<n$ satisfies $a^i\equiv1\pmod{m}$ (since $a^i-1$ is a positive integer smaller than $m$). \newline
  This implies that order of $H$ equals $n$
  $\therefore$ the order of a subgroup always divides the order of a group, $n\mid \phi(a^n-1)$
\end{solution}

\setcounter{problemname}{5}
\begin{problem}
  如果$n\geq 2$, 证明:$n\nmid 2^n -1$
\end{problem}

\begin{solution}
  Proof by Contradiction: \newline
  Assume that there is an integer $n\geq 2 \ni n \mid 2^n - 1$, clearly, $n$ is odd \newline
  Take $p$ to be the smallest prime dividing $p \mid 2^n -1, p \mid 2^{p-1} - 1$
  $$ \because gcd(a^k-1, a^l - 1) = a^{gcd(k,l)} - 1, k, l \in \mathbb{Z}^+, a > 1$$
  $$ \therefore p | 2^d -1, d = gcd(n, p -1)$$ 
  However, since $p$ is the smallest prime divisor of $n$ we have $d=1$. 
  Hence $p \mid 2^d - 1 = 1$ is a contradiction, thus $ n \nmid 2^n - 1$ 
\end{solution}

\setcounter{problemname}{6}
\begin{problem}
  设$p$是奇素数,$n\geq 1$,证明:
  \begin{equation}
    \sum^{p-1}_{k=1}{k^n} \equiv   
    \left\{\begin{array}{@{}l@{}}
     -1\pmod{p}, \text{如果} p-1 \mid n\\
     0\pmod{p}, \text{如果} p-1 \nmid n
    \end{array}\right.\,.
  \end{equation}
\end{problem}

\begin{solution}
  We can use the rule that every prime has a primitive root such that there is an $x$ and considering the set
  $$[x^1, x^2, \dots, x^{p-1}] \pmod{p} \equiv [1,2,\dots, (p-1)] \pmod{p}$$
  $$ \therefore 1^k + 2^k + \cdots + (p-1)^k = x^k + x^{2k} + \cdot + x^{(p-1)k} \pmod{p}$$
  Using the geometric sum, then
  $$ \Rightarrow (x^k)(1-x^{(p-1)k})/ (1-x^k)\pmod{p}$$ 
  Because of Fermats little theorem $ a^{p-1} \equiv 1 \pmod{p}$, $(1-x^{(p-1)k}) = 0$
  $$\therefore (x^k)(0)/ (1-x^k)\pmod{p} = 0 \pmod{p}$$ 
  If $p -1 \mid n$, then $n = (p-1)j$ for some $j$
  $$ 1^n+\cdots+(p-1)^n\equiv 1^{(p-1)j}+\cdots+(p-1)^{(p-1)j}\equiv 1+\cdots+1 = p - 1 \equiv -1\pmod{p} $$
  because If $p-1$ divides $n$, then by Fermat's Theorem each term is congruent to $1$ modulo $p$. There are $p-1$ terms, so the sum is congruent to $-1$ modulo $p$.
\end{solution}

\begin{problem} \newline
  $(1)$ 设$F_n = 2^{2^n} + 1$ (费马数), $n\geq 1$. 证明:$F_n$的每个素因子都有形式$2^{n+1}x + 1, x\in \mathbb{Z}$ \newline
  $(2)$ 对任意给定的整数$l\geq1$,证明:若无穷多个素数模$2^l$余$1$  
\end{problem}

\begin{solution} \newline
  $(1)$ Lemma: $p^2 \leftrightharpoons p \equiv \pm 1\pmod{8}$ \newline
  Let $p$ be a divisor of the Fermat Number $2^{2^n}+1$
  $$\therefore 2^{2^n}\equiv-1\pmod{p}$$
  $$\therefore (2^{2^n})^2\equiv 1\pmod{p}$$
  $$\therefore 2^{2^{(n+1)}} \equiv 1\pmod{p}$$
  So $x=2^{n+1}$ is $2^x \equiv1\pmod{p}$ smallest integer solution  \newline
  $$\therefore 2 \text{ 对模 } p \text{ 的指数是 } 2^{n+1}$$ \newline
  与费马小定理 $2^{p-1}\equiv 1\pmod{p}$ 比较得 $2^{n+1}\mid(p-1)$ \newline 
  当 $n>1$, $p\equiv 1\pmod{8}$, using the lemma, $2$ is $p$ square remainder \newline 
  $$\because 2^{(p-1)/2} \equiv 1\pmod{p}$$ \newline
  $\therefore 2^{n+1}| \frac{p-1}{2}$, 令$(p-1)/2=2^{n+1}*k$ 即得 $p=2^{n+2}*k+1$ \newline
  $(2)$
\end{solution}





\begin{problem} \newline
  $(1)$ 设$p$为奇素数,$a\geq2$. 证明:若$a^p-1$的素因子$q$不整除$a-1$则必有形式$q=2px+1, x\in\mathbb{Z}$ \newline
  $(2)$ 设$p$给定的奇素数,证明:形如$2px+1(x\in\mathbb{Z})$的素数有无限多个
\end{problem}

\begin{solution} \newline
  $(1)$
  $$ a^p - 1= (a-1) [a^{p-1} +a^{p-2} + \cdots + 1] $$
  由费马小定理
  $$ a^{p-1} \equiv 1 \pmod{p} $$ 
  $$ \therefore a^{p-1}=mp+1 $$ 
  $$ \Rightarrow [a^{p-2} + \cdots + 1 ] = [a^{p-1}-1]/(a-1) = mp/(a-1) $$ 
  所以 $[a^{p-2} + \cdots + 1]$ 有因数 $p$ \newline 
  $[a^{p-1} +a^{p-2}+\cdots+1]$ 共有 $p$项 即奇数项 \newline
  除去最后一项1 还有偶数项。无论a为奇数还是偶数 \newline
  $[a^{p-1} +a^{p-2}+\cdots+a]$ 均为偶数 \newline
  $\therefore [a^{p-1} +a^{p-2}+\cdots+1] = 2px + 1$ \newline 
  $\therefore a^{p}-1 =(a-1)(2px+1)$ \newline
  如果不整除$a-1$,必须整除$2px+1$ \newline
  $(2)$ 
\end{solution}   

\section{}
\setcounter{problemname}{0}
\begin{problem}
  证明:$m$是一个素数充分必要条件是存在$a$,$a$对模$m$的次数为$m-1$
\end{problem}

\begin{solution}
  If $a$ has order $m-1$, then by Euler's theorem $m - 1 \mid \phi(m)$ \newline
  This occurs when $m$ is prime, since $\phi(m) = m - 1$. Thus, $m$ is prime
\end{solution}

\begin{problem}
  设$g$是奇素数,$p$的一个原根,证明:\newline
  $(1)$ 当$p\equiv1\pmod{4}$时,$-g$也是$p$的一个原根\newline
  $(2)$ 当$p\equiv3\pmod{4}$时,$-g$对$p$的次数为$\frac{p-1}{2}$
  
\end{problem}

\begin{solution} \newline
  $(1)$ Since $p$ is odd, and we have Fermat's little theorem, 
  $$ \because a^p \equiv a \pmod{p} $$
  $$ \therefore g \equiv g^p \equiv -(-g)^p\pmod{p}$$
  Since $p\equiv 1\pmod{4}, x^2\equiv -1\pmod{p}$, -1 is a quadradic residual of p, ($-1$ is a square mod $p$ iff $p \equiv 1\pmod{4}$)
  $$ \exists k \in \mathbb{Z} \backepsilon -1 \equiv g^{2k}\equiv (-g)^{2k}\pmod{p}$$
  Thus $g\equiv (-g)^{2k} (-g)^{p} \pmod{p}$. Since $g$ is congruent to $-g^p$, $-g$ is also a primative root of p.
  $(2)$ Using a primative root principle, that \newline
  \begin{equation} 
    a \text{ is of order } h\pmod{n}\text{, then } a^k \text{ is of order } \frac{h}{gcd(h,k)} 
  \end{equation}
  Since $g$ is a primitive root, 
  $$-1\equiv g^{(p-1)/2}\pmod{p}$$ 
  $$ \therefore -g\equiv(-1)(g)\equiv g^{(p-1)/2}g\equiv g^{(p+1)/2}\pmod{p}$$ 
  Now, the order of $g^{(p+1)/2}\pmod{p}$ 
  according to (5) is $\frac{p-1}{gcd((p+1)/2,p-1)} $ \newline
  If $p\equiv 1\pmod{4}$, then $(p+1)/2$ is odd and $gcd((p+1)/2,p-1)$ is 1, making the order of $-g$ to be $p-1$, thus it is a primative root. \newline 
  Otherwise, the term $\frac{p+1}{2}$ is even and  $gcd(\frac{p+1}{2},p-1) = (p-1)/2>1$ \newline
  Therefore, the order of $-g$ is not $p-1$. i.e. not a primitive root.
\end{solution}

\vfill
\vspace{0.25\textheight}




\end{CJK}
\end{document}