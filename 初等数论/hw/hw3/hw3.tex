\documentclass[12pt, a4paper, oneside]{article}
\usepackage{CJKutf8}
\usepackage{amsmath, amsthm, amssymb, bm, color, framed, graphicx, hyperref, mathrsfs}
\usepackage{fancyhdr}
\usepackage[left=20mm, right=20mm, top=20mm, bottom=20mm]{geometry}
\pagestyle{fancy}
\fancyhf{}
\rhead{\begin{CJK}{UTF8}{gbsn}{计83 李天勤 2018080106}\end{CJK}}
\lhead{\begin{CJK}{UTF8}{gbsn}{初等数论及应用作业3}\end{CJK}}
	

\title{\textbf{课程作业}}
\author{Dylaaan}
\date{\today}
\newpage
\linespread{1.5}
\definecolor{shadecolor}{RGB}{241, 241, 255}
\newcounter{problemname}
\newenvironment{problem}{\begin{shaded}\stepcounter{problemname}\par\noindent\textbf{题目\arabic{problemname}. }}{\end{shaded}\par}
\newenvironment{solution}{\par\noindent\textbf{解答. }}{\par}
\newenvironment{note}{\par\noindent\textbf{题目\arabic{problemname}的注记. }}{\par}

\begin{document}
\begin{CJK}{UTF8}{gbsn}

% \maketitle
% \newpage

\section{2.1}


\begin{problem}
    设$m$位正整数,$(a,m)=1$. 我们用$a^{-1}$表示同余方程$ax\equiv1\pmod{m}$的任何一个整数解(即$a^{-1}\in\mathbb{Z}, aa^{-1}\equiv\pmod{m}$). 证明
\end{problem}

\begin{solution}
  
\end{solution}


\vfill
\vspace{0.25\textheight}

\begin{problem}
    设正整数$n$的十进制表示为 
    $$ n = a_0 + a_1 \cdot 10 + a_2 \cdot 10^2 + \cdots + a_k \cdot 10^k $$ 
    证明
\end{problem}

\begin{solution} 

\end{solution}

\vfill
\vspace{0.25\textheight}

\newpage


\setcounter{problemname}{5}
\begin{problem}
   (1) 证明:当$n\geq3$时,$\phi(n)$为偶数
\end{problem}

\begin{solution}

\end{solution}

\vfill
\vspace{0.25\textheight}



\begin{problem}
    设$m$和$n$是正整数,$n=nt(t\in\mathbb{Z})$. 证明:模$n$的每个同余类都是模$m$的$t$个同余类之并。
\end{problem}

\begin{solution}

\end{solution}

\vfill
\vspace{0.25\textheight}
\newpage


\setcounter{problemname}{9}
\begin{problem}
  设$a,m\in\mathbb{Z}, m \geq 2, (a,m) = 1$, 计算 
  $$ \sum^{m-1}_{x=0}{[\frac{ax}{m}]}$$
\end{problem}

\begin{solution}
  
\end{solution}

\newpage



% \vfill
% \vspace{0.3\textheight}



% \vfill
% \vspace{0.3\textheight}

% \newpage



\section{2.2}

\setcounter{problemname}{0}
\begin{problem}
  设$a$是环$\mathbb{Z}_m$中非零元素,如果存在$\mathbb{Z}_m$中非零元素$\beta(\neq 0')$, 使得 $\alpha\beta=0'$, 称$a$零因子,证明
\end{problem}

\begin{solution}

\end{solution}   

\vfill
\vspace{0.25\textheight}

\begin{problem} \newline
  $(1)$ 对与环$\mathbb{Z}_m$中任何元素$\alpha, m$个$\alpha$相加为$0'$ \newline
  $(2)$ 设$p$为素数,对于域$\mathbb{Z_p}$中非零元素$\alpha$和正整数$n$,证明:$n$个$\alpha$相加为$0'$当且仅当$p \mid n$
\end{problem}

\begin{solution}

\end{solution}

\vfill
\vspace{0.25\textheight}
\newpage

\begin{problem}
  证明当$p$为奇素数时
\end{problem}

\begin{solution}
  
\end{solution}

\vfill
\vspace{0.25\textheight}

\begin{problem}
  对于整数$m\geq2$证明:$(m-1)!\equiv-1\pmod(m)$当且仅当$m$时素数
\end{problem}

\begin{solution}
  
\end{solution}

\vfill
\vspace{0.25\textheight}
\newpage

\begin{problem}
  证明,若$Z_m^\star={a_1,\dots,a_\phi(m)}$,则$Z_m^\star={a_{1}^{-1},\dots,a_{\phi(m)}^{-1}}$.如何将它转述成同于的语言?
\end{problem}

\begin{solution}
  
\end{solution}


\end{CJK}
\end{document}